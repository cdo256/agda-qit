% Options for packages loaded elsewhere
\PassOptionsToPackage{unicode}{hyperref}
\PassOptionsToPackage{hyphens}{url}
\documentclass[a4paper,UKenglish,cleveref, autoref, thm-restate]{lipics-v2021}
\usepackage{mathpartir}
\usepackage{amsmath}
\bibliographystyle{plainurl}

\providecommand{\tightlist}{%
  \setlength{\itemsep}{0pt}\setlength{\parskip}{0pt}}

\title{Constructing QITs from Quotients}

\author{Thorsten Altenkirch}{University of Nottingham, UK}{thorsten.altenkirch@nottingham.ac.uk}{https://orcid.org/0000-0002-6582-5025}{}
\author{Christina O'Donnell}{University of Nottingham, UK}{psxco3@nottingham.ac.uk}{[ORCID]}{Supported by the Engineering and Physical Sciences Research Council (EPSRC) Doctoral Landscape Award [Grant Ref: EP/YXXXXXX/1].}

\authorrunning{T. Altenkirch and C. D. O'Donnell}
\Copyright{Thorsten Altenkirch and Christina O'Donnell}

\ccsdesc[500]{Theory of computation~Type theory}
\ccsdesc[300]{Theory of computation~Constructive mathematics}
\ccsdesc[300]{Theory of computation~Logic and verification}

\keywords{Quotient Inductive Types, W-Types, Homotopy Type Theory, Agda, Constructive Mathematics}

% Optional: Link to the ArXiv version or repository if applicable
%\relatedversion{Full version hosted on arXiv: \url{https://arxiv.org/abs/xxxx.xxxxx}}

\supplement{Source code available at \url{https://github.com/cdo256/agda-qit}}

\begin{document}

\maketitle

\section{Motivation}\label{motivation}

Quotient Inductive Types (QITs) are a central mechanism for specifying
datatypes equipped with identifications in set-truncated Homotopy Type
Theory \cite{hott2013-book}. They support a wide range of fundamental
constructions, including the HoTT reals \cite{hott2013-book}, partiality
monads \cite{altenkirch2017-partiality-monad}, and ordinal-like
structures \footnote{TODO: Citaiton}.

Despite their expressive power, QITs present significant foundational
difficulties. Shulman and Lumsdaine established a no-go theorem showing
that certain QITs, including Brower's standard definition of the countable
ordinals, cannot be constructed using quotients alone
\cite{lumsdaine2020-semantics-higher-inductive}. This suggests that some
form of infinitary principle or choice is unavoidable in a constructive
metatheory, at least in the general case.

Fiore, Pitts, and Steenkamp subsequently showed that a broad class of
QITs, which they call Quotient-W Types (QW types), admit an Initial
Algebra semantics assuming the Weak Initial Set of Covers principle (WISC)
\cite{fiore2022-quotients-inductive-types}. Their result
recovers many important examples, including the extensional countable
ordinals. However, WISC remains a non-constructive choice principle
\cite{vandenberg2012-wisc-axiom}, and its necessity is poorly
understood. In particular, it is unclear which features of a QIT
genuinely require choice, and which arise from limitations of existing
proof techniques.

This leads to our central research question:

\begin{quote}
	Which quotient-W types can be constructed fully constructively, without
	assuming any form of choice?
\end{quote}

% Say somewhere WISC is not a choice principle
% Breadth is irrelevant, it's only a global depth bound needed to compare an arbitrary pair of nodes.

%TODO: Define point and path constructor in general?
%TODO: Reword?
Our approach is to study one of the simplest infinitary previously thought to
require choice example to test the need for WISC in defining QITs, while still
being as simple as possible. We then try
to prove that these exhibit an initial QW-algebra---a categorical structure that
represents that there is exists a recursor (morphism between algebras), and that
that recursor is unique (no `extra stuff' in Q beyond what is needed to
represent the signature). We formalised this example in Agda and traced
precisely where WISC is used in the Fiore--Pitts--Steenkamp construction to see
if we could determine a weaker principle, that was more justified
constructively. What we discovered, to our surprise, is that the step that Fiore
et al. used WISC could actually be shown to be fully constructive. We postulate
that a large class of QITs can be exhibited constructively.

%TODO: Define WISC?

\section{Mobiles}
The example that we chose--`mobiles'--have just two point constructors and one
path constructor, making it much simpler than the other 'hard' cases we
surveyed, while still being considered as challenging from a constructability
standpoint.

We define *mobiles* as well-founded trees quotiented by permutation at each
level. (TODO: Figures). Specifically,

$$
\begin{array}{l@{\; }l@{\quad}l}
  \multicolumn{3}{l}{\textbf{data } \mathsf{Mobile} : \mathcal{U} \textbf{ where}} \\
  \quad \mathsf{leaf}        & : \mathsf{Mobile} \\
  \quad \mathsf{node}(f)     & : \mathsf{Mobile}                                  & \forall f : \mathsf{Mobile}^I \\
  \quad \mathsf{perm}(\pi,f) & : \mathsf{node}(f) = \mathsf{node}(f \circ \pi) & \forall \pi : I \cong I ;\, f : \mathsf{Mobile}^I
\end{array}
$$

for some $I:\mathcal{U}$ where $\mathsf{Mobile}^I$ represents functions from the
branching index set $I$ and $\mathsf{Mobile}$'s. This has a clean, natural
presentation that fits in three lines, unlike others which require auxillary
data types, or stronger recursion semantics. %TODO: Terminology?

\section{Main Idea}\label{main-idea}

%Mention

We briefly recall the framework of quotient-W types introduced by Fiore,
Pitts, and Steenkamp \cite{fiore2022-quotients-inductive-types}. A QW
signature consists of a container (representing branching structure) and a set
of equations over the container. From such a signature, Fiore et al.~construct a diagram of
approximations indexed by a size structure
\cite{fiore2022-quotients-inductive-types}. Intuitively, this diagram
represents successive stages of the quotient construction, where each
stage gives the proofs available within a bounded depth. The desired QW
type is obtained as the colimit $Q$ of this diagram.

Fiore et al.~show this that this diagram is in fact an initial algebra: having a
unique recursor, thereby showing that the structure using (indexed) WISC
\cite{fiore2022-quotients-inductive-types}. In particular,
WISC is crucial in proving \emph{cocontinuity}, meaning that forming
constructors commute with quotienting, and that $Q$ is closed under all
constructors.

The proof in Fiore et al.~relies on WISC to select a cover to size-bound the
family of subtrees, enabling the construction of inverse maps witnessing
this isomorphism.

Our work revisits this step. We observe that the need for WISC arises
precisely when the equations allow arbitrarily deep inspection of terms,
as in the extensional ordinal example. When all equations are
depth-bounded, the required coherence data can instead be constructed by
well-founded recursion on a fixed ordinal bound \(\alpha\), avoiding any
appeal to choice. As a consequence, we obtain a fully constructive construction
of infinite mobiles.

We do this by introducing a depth-bound needed to compare any two values, by
making use of the fact that every equation preserves tree depths, so if a pair
of trees are equal then they must be at the same depth, and they must be
comparable without intermediate terms of unbounded depth
\cite{pitts2021-inflationary-iteration}.

\section{Conclusion}\label{conclusion}

We identify bounded locality as a structural criterion characterising
which quotient-W types admit fully constructive constructions. This
explains when choice principles such as WISC are genuinely necessary,
and when they can be avoided. Our results show that many infinitary
quotients previously thought to require choice are, in fact,
constructible in a purely constructive setting.

Our work builds on the QW framework of Fiore, Pitts, and Steenkamp
\cite{fiore2022-quotients-inductive-types}, and complements the no-go
results of Shulman and Lumsdaine
\cite{lumsdaine2020-semantics-higher-inductive}. It is also related to
Pitts and Steenkamp's notion of infinitary inflation
\cite{steenkamp2021-phd-thesis}, which similarly isolates sources of
non-constructivity in inductive definitions.

Future directions include formalising the locality theorem stated above,
and extending the analysis to indexed QW types
\cite{altenkirch2018-quotient-inductive-inductive}.

\bibliography{master}

\end{document}
