\documentclass{easychair}

\usepackage{amsmath}
\usepackage{amssymb}
\usepackage{mathpartir}
\usepackage{graphicx}

\providecommand{\tightlist}{%
\setlength{\itemsep}{0pt}\setlength{\parskip}{0pt}}

\title{Constructing QITs from Quotients}

\author{
    Thorsten Altenkirch\inst{1}
\and
    Christina O'Donnell\inst{1}\thanks{Supported by the Engineering and Physical Sciences Research Council (EPSRC) Doctoral Landscape Award [Grant Ref: EP/Z534948/1].}
}

\institute{
    University of Nottingham, UK\\
    \email{\{psztxa,psxco3\}@nottingham.ac.uk}
}

\authorrunning{T. Altenkirch and C. D. O'Donnell}
\titlerunning{Constructing QITs from Quotients}

\begin{document}

\maketitle

\section{Motivation}\label{motivation}

Quotient Inductive Types (QITs) are a central mechanism for specifying
datatypes equipped with identifications in set-truncated Homotopy Type
Theory~\cite{hott2013-book}. They support a wide range of fundamental
constructions, including the HoTT reals~\cite{hott2013-book}, partiality
monads~\cite{altenkirch2017-partiality-monad}, and various presentations of
ordinals. However, QITs present significant foundational
difficulties. Shulman and Lumsdaine established a no-go theorem showing
that certain QITs, including Brouwer's standard definition of the
countable ordinals, cannot be constructed using quotients alone
\cite{lumsdaine2020-semantics-higher-inductive}. This suggests that some
form of infinitary principle or choice is unavoidable in a constructive
metatheory, at least in the general case.

Fiore, Pitts, and Steenkamp subsequently showed that a broad class of
QITs, which they call Quotient-W Types (QW types), admit an initial
algebra semantics assuming the Weak Initial Set of Covers principle
(WISC)~\cite{fiore2022-quotients-inductive-types}. Their result recovers
many important examples, including the extensional countable ordinals.
However, WISC, although constructively acceptable, is not well aligned with the
internal, computational character of type theory,
\cite{vandenberg2012-wisc-axiom}, and its necessity is poorly
understood. In particular, it is unclear which features of a QIT
genuinely require choice, and which arise from limitations of existing
proof techniques. We therefore ask:

\begin{quote}
Which QITs can be constructed without WISC?
\end{quote}

Our approach is to study one of the simplest infinitary examples
previously thought to require choice, in order to isolate the essential
use of WISC.

We formalised this example in Agda and traced precisely where WISC is used in
the Fiore–Pitts–Steenkamp construction. To our surprise, the step where WISC is
invoked can in fact be carried out fully constructively. We therefore postulate
that a significant large class of infinitary QITs can be constructed from
quotient types only. 

\section{Mobiles}

We study \emph{mobiles}, which has just two point constructors
and one path constructor, making it substantially simpler than other
challenging cases, while still being considered difficult from a
constructability standpoint.

We define mobiles as well-founded trees quotiented by permutation at
each level. Concretely,

$$
\begin{aligned}
  &\textbf{data } \mathsf{Mobile}(I : \mathcal{U}) : \mathcal{U} \textbf{ where} \\
  &\quad \mathsf{leaf} : \mathsf{Mobile}(I) \\
  &\quad \mathsf{node} : (I \to \mathsf{Mobile}(I)) \to \mathsf{Mobile}(I) \\
  &\quad \mathsf{perm} : (\phi : \mathsf{Iso}(I, I)) \to (f : I \to \mathsf{Mobile}(I)) \to \mathsf{node}(f) = \mathsf{node}(f \circ \phi)
\end{aligned}
$$

for some $I : \mathcal{U}$, which may be infinite, where $\mathsf{Mobile}^I$
denotes functions from the branching index set $I$ to $\mathsf{Mobile}$. This
definition has a clean and compact presentation, in contrast to other examples
which require auxiliary datatypes or stronger recursion principles.

Unlike ordinal constructions, permutation does not involve recursive
inspection of subtrees.

\section{Main Idea}\label{main-idea}

We briefly recall the framework of quotient-W types introduced by Fiore et al.
~\cite{fiore2022-quotients-inductive-types}. A QIT
signature consists of a tree branching structure and a set of equations over
that shape. From such a signature, they construct a diagram of
approximations indexed by a size structure. Intuitively, this diagram represents
successive stages of the quotient construction, where each stage captures the
proofs available within a bounded depth. The desired QIT type is obtained as the
colimit $Q$ of this diagram.

Fiore et al.~show that this diagram yields an initial algebra, hence
supporting a unique recursor. Their proof relies essentially on WISC,
notably to establish \emph{cocontinuity}, meaning that forming
constructors commutes with quotienting and that $Q$ is closed under all
constructors.

Concretely, WISC is used to select covers that bound the family of
subtrees, enabling the construction of inverse maps witnessing the
required isomorphism.

We observe that the need for WISC arises
precisely when equations allow arbitrarily deep inspection of terms, as
in the extensional ordinal example. When all equations are
\emph{depth-bounded}, the required coherence data can instead be
constructed by well-founded recursion on a fixed ordinal bound
$\alpha$, avoiding any appeal to choice. As a consequence, we obtain a
WISC-free construction of infinite mobiles.

We exploit the fact that every defining equation for mobiles preserves
tree depth. Thus, if two trees are equal, they must occur at the same
depth, and can be compared without traversing intermediate terms of
unbounded depth~\cite{pitts2021-inflationary-iteration}.

\section{Conclusion}\label{conclusion}

We show that it is possible for non-trivial infinitary QITs to admit an initial
algebra without appealing to WISC. In particular, we show that infinite mobiles,
previously believed to require choice, admit a purely constructive construction.
This suggests that there may be a sigificant class of QITs which were previously
thought to require choice, but which can in fact be expressed without any form
of choice.


Our work builds on the QIT framework of Fiore et al.
\cite{fiore2022-quotients-inductive-types}, and complements the no-go results of
Shulman and Lumsdaine \cite{lumsdaine2020-semantics-higher-inductive}. It is
also related to Pitts and Steenkamp's notion of infinitary inflation
\cite{steenkamp2021-phd-thesis}, which similarly isolates sources of
non-constructivity in inductive definitions.

An immediate next step is to identify a \emph{syntactic} criterion ensuring that
a given QIT can be expressed without relying on choice principles. We propose a
notion of \emph{locality}, placing a global ordinal depth bound on all
generating equations. We speculate that such a bound would suffice to show that
any equality proof need only consider expansions up to an ordinal of the form
$\alpha \cdot \omega$, and hence that the QIT admits a fully constructive semantics.

Future work includes formalising this locality criterion, extending the analysis
to a broader class of QITs \cite{altenkirch2018-quotient-inductive-inductive}, and
developing a syntax-directed method for identifying constructible QITs.

\bibliographystyle{plain}
\bibliography{master}

\end{document}
