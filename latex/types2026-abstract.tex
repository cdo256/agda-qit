% Options for packages loaded elsewhere
\PassOptionsToPackage{unicode}{hyperref}
\PassOptionsToPackage{hyphens}{url}
\documentclass[a4paper,UKenglish,cleveref, autoref, thm-restate]{lipics-v2021}
\bibliographystyle{plainurl}

\providecommand{\tightlist}{%
  \setlength{\itemsep}{0pt}\setlength{\parskip}{0pt}}

\title{Constructing QITs from Quotients}

\author{Thorsten Altenkirch}{University of Nottingham, UK}{thorsten.altenkirch@nottingham.ac.uk}{https://orcid.org/0000-0002-6582-5025}{}
\author{Christina O'Donnell}{University of Nottingham, UK}{psxco3@nottingham.ac.uk}{[ORCID]}{Supported by the Engineering and Physical Sciences Research Council (EPSRC) Doctoral Landscape Award [Grant Ref: EP/YXXXXXX/1].}

\authorrunning{T. Altenkirch and C. D. O'Donnell}
\Copyright{Thorsten Altenkirch and Christina O'Donnell}

\ccsdesc[500]{Theory of computation~Type theory}
\ccsdesc[300]{Theory of computation~Constructive mathematics}
\ccsdesc[300]{Theory of computation~Logic and verification}

\keywords{Quotient Inductive Types, W-Types, Homotopy Type Theory, Agda, Constructive Mathematics}

% Optional: Link to the ArXiv version or repository if applicable
%\relatedversion{Full version hosted on arXiv: \url{https://arxiv.org/abs/xxxx.xxxxx}}

\supplement{Source code available at \url{https://github.com/cdo256/agda-qit}}

\begin{document}

\maketitle

\section{Introduction and Motivation}\label{introduction-and-motivation}

Quotient Inductive Types (QITs) are a central mechanism for specifying
datatypes equipped with identifications in set-truncated Homotopy Type
Theory \cite{hott2013-book}. They support a wide range of fundamental
constructions, including the HoTT reals \cite{hott2013-book}, partiality
monads \cite{altenkirch2017-partiality-monad}, and ordinal-like
structures \footnote{TODO: Citaiton}.

Despite their expressive power, QITs present significant foundational
difficulties. Shulman and Lumsdaine established a no-go theorem showing
that certain QITs, including Brower's definition of the countable ordinals,
cannot be constructed using quotients alone
\cite{lumsdaine2020-semantics-higher-inductive}. This suggests that some
form of infinitary principle or choice is unavoidable in a constructive
metatheory.

Fiore, Pitts, and Steenkamp subsequently showed that a broad class of
QITs, which they call Quotient-W Types (QW types), admit an Initial
Algebra semantics assuming the Weak Initial Set of Covers (WISC)
principle \cite{fiore2022-quotients-inductive-types}. Their result
recovers many important examples, including the extensional countable
ordinals. However, WISC remains a nontrivial choice principle
\cite{vandenberg2012-wisc-axiom}, and its necessity is poorly
understood. In particular, it is unclear which features of a QIT
genuinely require choice, and which arise from limitations of existing
proof techniques.

In this work we isolate two structural properties of quotient systems
that explains this distinction: \emph{locality} and the tighter property
of \emph{depth-delta}. Intuitively, locality measures how deeply one
must inspect the inductive structure of terms in order to witness that
two elements are equal while depth-delta measures the maximum depth
change an equation. In pathological examples, such as tree ordinals, the
principle of extensionality quantifies over \emph{all} ordinals less
than a pair of ordinals, unbounded descent into subtrees, forcing the
use of global choice principles. By contrast, in examples such as
infinite mobiles---infinitely branching trees quotiented by
permutation---equality preserves depth and never requires inspecting
substructure beyond a fixed finite level. In fact because permutation
doesn't change the subtree relation, we can say that it is in a class of
depth-preserving QW types, along with the classical definition of
multi-sets as a QW type over a list.

Our key observation is that this boundedness of inspection depth is
exactly what is needed to recover the Fiore--Pitts--Steenkamp
construction constructively. We sketch a proof that for QW signatures
whose equations are bounded by a fixed ordinal \(\alpha\), cocontinuity
of the associated polynomial functor can be proved without any appeal to
any choice principle. As a consequence, we obtain a fully constructive
construction of infinite mobiles as an initial QIT, contrary to previous
expectations, which we have formalized in Agda
\cite{vezzosi2021-cubical-agda-dependently}.

This leads to our central research question:

\begin{quote}
	Which quotient-W types can be constructed fully constructively, without
	assuming any form of choice?
\end{quote}

We propose bounded locality as a criterion separating constructible QITs
from genuinely non-constructive ones, and we develop a framework for
expressing and exploiting this boundedness in the construction of
initial QW-algebras.

\section{Background}\label{background}

We briefly recall the framework of quotient-W types introduced by Fiore,
Pitts, and Steenkamp \cite{fiore2022-quotients-inductive-types}. A QW
signature consists of:

\begin{itemize}
	\item
	      A container \(\Sigma = S \triangleleft P\)
	      \cite{abbott2005-containers}, specifying the shape and positions of a
	      Polynomial Functor whose initial algebra is a W-Type
	      \cite{martinlof1984-intuitionistic}.
	\item
	      A system of equations \((E, V, l, r)\), where each equation \(e : E\)
	      has a set of variables \(V(e)\) and left- and right-hand sides given
	      by terms in the free \(\Sigma\)-algebra over those variables.
\end{itemize}

From such a signature, Fiore et al.~construct a diagram of
approximations indexed by a size structure
\cite{fiore2022-quotients-inductive-types}. Intuitively, this diagram
represents successive stages of the quotient construction, where each
stage gives the proofs available within a bounded depth. The desired QW
type is obtained as the Colimit of this diagram.

To show that this colimit yields an initial algebra, two key properties
are required:

\begin{enumerate}
	\def\labelenumi{\arabic{enumi}.}
	\tightlist
	\item
	      The existence of suitable covers for the underlying W-type and the
	      equation contexts.
	\item
	      \emph{Cocontinuity} of the associated polynomial functor,
	      i.e.~preservation of the colimit.
\end{enumerate}

Both properties are established in Fiore et al.~using the indexed WISC
principle \cite{fiore2022-quotients-inductive-types}. In particular,
WISC is crucial in proving cocontinuity, which amounts to constructing a
isomorphism for the functor \(F\) determined by the container
\(\Sigma\).

Operationally, cocontinuity expresses the fact that applying
constructors commutes with taking quotients: forming a node after
quotienting is equivalent to quotienting after forming a node. The proof
in Fiore et al.~relies on WISC to select a cover to size-bound the
family of subtrees, enabling the construction of inverse maps witnessing
this isomorphism.

Our work revisits this step. We observe that the need for WISC arises
precisely when the equations allow arbitrarily deep inspection of terms,
as in the extensional ordinal example. When all equations are
depth-bounded, the required coherence data can instead be constructed by
well-founded recursion on a fixed ordinal bound \(\alpha\), avoiding any
appeal to choice.

\section{Locality Principle}\label{locality-principle}

We formalise the notion of locality by assigning a \emph{rank} function

\[ \mathrm{rank} : T_\Sigma X \to \mathcal{O} \]

measuring the height of the underlying tree. An equation system is said
to be \(\alpha\)-local if, for every equation \(e : E\), both sides
\(l(e)\) and \(r(e)\) have rank \emph{at most} a fixed ordinal
\(\alpha\). Intuitively, this means that every generating equation only
inspects structure up to depth \(\le \alpha\).

\[ \mathrm{rank} : T_\Sigma X \to \mathcal{O} \]

\section{Main Result}\label{main-result}

Our main theorem is that \(\alpha\)-local QW signatures admit fully
constructive initial algebra semantics.

\emph{Theorem} Let \(\Sigma\) be a QW signature whose equations are
\(\alpha\)-local. Then the associated quotient-W type exists as an
initial algebra, and its construction does not require WISC or any other
choice principle.

The proof follows the outline of Fiore--Pitts--Steenkamp
\cite{fiore2022-quotients-inductive-types}, but replaces the use of WISC
in the inverse map by offsetting by a bound ordinal \(\alpha\), giving a
constructable function between term bound and maximum proof depth,
giving cocontinuity. This construction is fully constructive and does
not require any choice principles.

\section{Case Study: Infinite
  Mobiles}\label{case-study-infinite-mobiles}

Our formalisation in Agda proves that in the case of mobiles, the
defining equations are provably \emph{depth-preserving}, since all
variables on the left appear at the same depth on the right. This is
enough to show that depth is preserved in the quotient type, and that if
two terms are equal then they can be proven in the stage bounded by
either ordinal.

As a result, the main theorem yields a fully constructive construction
of infinite mobiles as an initial QIT. This provides a concrete example
of a genuinely infinitary quotient type that does \emph{not} require
choice, contrary to previous assumptions.

\section{Boundary Phenomena}\label{boundary-phenomena}

Our analysis explains why certain examples genuinely require choice. In
the case of extensional ordinals, equality is defined by mutual
containment of downward closures, which requires unbounded descent into
subtrees \cite{fiore2022-quotients-inductive-types}. No ordinal bound
\(\alpha\) suffices, and the quotient is therefore non-local. This
accounts for the essential use of WISC in the construction of ordinals.

More generally, any quotient whose generators quantify over arbitrarily
deep substructures or require unbounded transitive closure will fall
outside the \(\alpha\)-local fragment.

\section{Method Overview}\label{method-overview}

Technically, we construct stratified diagrams indexed by ordinals below
\(\alpha\), and show that all cones factor through a bounded stage.
Plumpness of the ordinal ensures the existence of suprema for such
cones, yielding cocontinuity without choice.

\section{Related Work}\label{related-work}

Our work builds on the QW framework of Fiore, Pitts, and Steenkamp
\cite{fiore2022-quotients-inductive-types}, and complements the no-go
results of Shulman and Lumsdaine
\cite{lumsdaine2020-semantics-higher-inductive}. It is also related to
Pitts and Steenkamp's notion of infinitary inflation
\cite{steenkamp2021-phd-thesis}, which similarly isolates sources of
non-constructivity in inductive definitions.

\section{Future Work}\label{future-work}

Future directions include formalising the locality theorem stated above,
and extending the analysis to indexed QW types
\cite{altenkirch2018-quotient-inductive-inductive}.

\section{Conclusion}\label{conclusion}

We identify bounded locality as a structural criterion characterising
which quotient-W types admit fully constructive constructions. This
explains when choice principles such as WISC are genuinely necessary,
and when they can be avoided. Our results show that many infinitary
quotients previously thought to require choice are, in fact,
constructible in a purely constructive setting.

\bibliography{master}

\end{document}
