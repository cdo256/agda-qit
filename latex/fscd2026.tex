\PassOptionsToPackage{unicode}{hyperref}
\PassOptionsToPackage{hyphens}{url}
\documentclass[a4paper,UKenglish,cleveref, autoref, thm-restate]{lipics-v2021}
\usepackage{mathpartir}
\usepackage{amsmath}
\bibliographystyle{plainurl}

\providecommand{\tightlist}{%
\setlength{\itemsep}{0pt}\setlength{\parskip}{0pt}}

\title{Constructing QITs from Quotients}

\author{Thorsten Altenkirch}{University of Nottingham, UK}{thorsten.altenkirch@nottingham.ac.uk}{https://orcid.org/0000-0002-6582-5025}{}
\author{Christina O'Donnell}{University of Nottingham, UK}{psxco3@nottingham.ac.uk}{}{Supported by the Engineering and Physical Sciences Research Council (EPSRC) Doctoral Landscape Award [Grant Ref: EP/Z534948/1].}

\authorrunning{T. Altenkirch and C. D. O'Donnell}
\Copyright{Thorsten Altenkirch and Christina O'Donnell}

\ccsdesc[500]{Theory of computation~Type theory}
\ccsdesc[300]{Theory of computation~Constructive mathematics}
\ccsdesc[300]{Theory of computation~Logic and verification}

\keywords{Quotient Inductive Types, W-Types, Homotopy Type Theory, Agda, Constructive Mathematics}

% Optional: Link to the ArXiv version or repository if applicable
%\relatedversion{Full version hosted on arXiv: \url{https://arxiv.org/abs/xxxx.xxxxx}}

\supplement{Source code available at \url{https://github.com/cdo256/agda-qit}}

\begin{document}

\maketitle

{
	\setcounter{tocdepth}{3}
	\tableofcontents
}
\section{Introduction}\label{introduction}

Quotient inductive types (QITs) extend ordinary inductive types with
equational identifications and provide a uniform way to define datatypes
equipped with canonical equalities. They play an important role in
formalised mathematics in type theory, where they are used to present
structures such as Cauchy reals, partiality monads, and various
constructions of ordinals. Despite their usefulness, QITs are known to
pose subtle foundational and semantic difficulties, especially when
their constructors involve infinitary branching or when their path
constructors express extensionality principles.

Fiore, Pitts, and Steenkamp (FPS) introduced a general semantic
framework for a large class of QITs, called quotient--W types (QW
types). Their construction proceeds by building a size-indexed diagram
of stagewise approximations and taking a colimit, yielding an initial
algebra that supports the expected elimination principle. However, their
proof that constructor formation commutes with this colimit relies on
the Weak Initial Set of Covers principle (WISC), a weak choice principle
that is not derivable in standard predicative type theory. While WISC is
constructively acceptable in many settings, its use raises a natural
question: which features of a QIT genuinely require such principles, and
which arise from the generality of the FPS argument?

In this paper we investigate this question through a concrete infinitary
example: \textbf{mobiles}, i.e.~well-founded trees with countably many
children at each node, quotiented by permutation of subtrees. Mobiles
form a conceptually simple QIT with one infinitary constructor and one
path constructor expressing that the order of children is irrelevant.
Mobiles form a conceptually simple QIT with one infinitary constructor
and one path constructor expressing that the order of children is
irrelevant. In earlier work of Fiore, Pitts, and Steenkamp, this example
is presented as a representative infinitary quotient inductive type for
which their general construction relies on WISC.
\cite{fiore2020-qw-types}

Our main result is that mobiles admit an initial algebra semantics
\textbf{without using WISC}. We achieve this by refining the FPS
size-indexed construction. We observe that the reliance on WISC in the
FPS proof is tied to equations that require recursive inspection at
unbounded depth, as in extensional definitions of ordinals. In contrast,
the permutation equations for mobiles are \emph{depth-bounded}: they act
only at a single tree level and do not require comparing subtrees
recursively. We formalise a corresponding locality condition on QIT
signatures and show that, under this condition, the key cocontinuity
step in the FPS construction can be proved by well-founded recursion on
the size index alone.

Concretely, we define mobiles as a quotient of well-founded
\(\mathbb{N}\)-branching trees by levelwise permutation and construct
them as the colimit of a transfinite sequence of stagewise quotients. We
show that constructor formation preserves this colimit constructively,
and derive the expected initial algebra property. All results have been
formalised in Agda.

This identifies a new subclass of infinitary QITs that can be
constructed purely from quotient types, without additional choice
principles. It also clarifies the role of WISC in the FPS framework:
choice is needed when equations enforce global, unbounded recursive
comparison, but not when they are local to bounded depth.

\subsection{Contributions}\label{contributions}

\begin{itemize}
	\item
	      \textbf{Constructive construction of an infinitary QIT.} We give a
	      construction of the quotient inductive type of infinitary mobiles
	      (well-founded countably branching trees modulo permutation) that does
	      not rely on WISC or any other choice principle.
	\item
	      \textbf{Refinement of the Fiore--Pitts--Steenkamp framework.} We
	      analyse the size-indexed QW-type construction of Fiore, Pitts, and
	      Steenkamp and isolate the precise step where WISC is used, namely the
	      cocontinuity argument showing that constructor formation commutes with
	      the transfinite colimit of stagewise quotients.
	\item
	      \textbf{Depth-bounded (local) equations as a constructivity
		      criterion.} We identify a structural condition on QIT signatures,
	      formulated as depth-boundedness of equations, under which the
	      cocontinuity step can be proved constructively by well-founded
	      recursion on the size index.
	\item
	      \textbf{Application to mobiles.} We show that the permutation
	      equations for mobiles satisfy the depth-boundedness condition, and
	      therefore that the FPS construction yields an initial algebra for
	      mobiles without invoking WISC.
	\item
	      \textbf{Mechanised development.} The construction and main
	      metatheoretic arguments have been formalised in Agda, providing
	      machine-checked evidence for the correctness of the development.
	\item
	      \textbf{Conceptual separation between local and global equations.} Our
	      analysis explains the contrast between mobiles and extensional ordinal
	      constructions: the former involve only local (levelwise) equations,
	      while the latter require unbounded recursive comparison, clarifying
	      why choice principles arise in one case but not the other.
\end{itemize}

\section{Background}\label{background}

\subsection{Inductive and Quotient Inductive
	Types}\label{inductive-and-quotient-inductive-types}

Inductive types in dependent type theory are characterised by initial
algebra semantics for strictly positive endofunctors and admit
introduction, elimination, and computation rules justified by
well-founded recursion
\cite{martinlof1984-intuitionistic,dybjer1994-inductive-families}.
Categorically, such types correspond to initial algebras of polynomial
functors, whose structure can be described concretely using containers
\cite{abbott2005-containers} and, more generally, dependent polynomial
functors \cite{gambino2004-wellfounded-trees-polynomial}. These
semantics extend to constructive set-theoretic settings via well-founded
trees
\cite{moerdijk2000-wellfounded-trees,aczel1978-type-theoretic-interpretation,palmgren2002-developments-cst}.

Quotient inductive types (QITs) extend ordinary inductive types by
including path constructors that impose equations between generated
elements. They are typically considered in set-truncated settings, where
all higher paths are forced to be trivial
\cite{awodey2012-type-theory-homotopy,kraus2015-truncation-levels}. QITs
have been used to give intrinsic presentations of structures such as
partiality monads \cite{altenkirch2017-partiality-monad} and appear as a
special case of the more general quotient inductive-inductive types
(QIITs) \cite{altenkirch2018-quotient-inductive-inductive}.

Despite their expressive power, QITs do not in general admit a simple
initial algebra semantics. Lumsdaine and Shulman showed that certain
higher inductive types cannot be constructed in standard model
categories without additional assumptions
\cite{lumsdaine2020-semantics-higher-inductive}. Even in the
set-truncated case, equations that express extensionality over
infinitary data can obstruct straightforward constructions.

\subsubsection{QW Types and Size-Indexed
	Constructions}\label{qw-types-and-size-indexed-constructions}

Fiore, Pitts, and Steenkamp introduced \emph{quotient--W types} (QW
types) as a general semantic framework for a large class of QITs
\cite{fiore2020-qw-types,fiore2022-quotients-inductive-types,steenkamp2021-phd-thesis}.
A QW signature consists of a polynomial branching structure together
with a family of equations between terms. From such a signature they
construct a diagram of stagewise approximations indexed by a
well-founded size structure. At each stage, one forms a quotient of
well-founded trees whose subtrees come from earlier stages. The desired
QIT is obtained as the colimit of this diagram.

The key technical step in their development is a \emph{cocontinuity}
result: the polynomial functor determined by the signature must preserve
the size-indexed colimit. This ensures that the colimit carries an
algebra structure and satisfies the expected initiality property. In the
general case, the proof of cocontinuity relies on the Weak Initial Set
of Covers (WISC) axiom \cite{vandenberg2012-wisc-axiom}, a weak choice
principle that guarantees the existence of sufficiently small families
of covers. Intuitively, WISC is used to bound the branching behaviour of
families of subtrees so that constructor formation can be exchanged with
the colimit.

Related ideas appear in Pitts and Steenkamp's work on inflationary
iteration, where initial algebras for certain infinitary constructions
are obtained via transfinite iteration under boundedness assumptions
\cite{pitts2021-inflationary-iteration}. In both settings, well-founded
size indices play a role analogous to ordinal stages in classical
constructions of inductive definitions.

\subsubsection{Choice Principles and Infinitary Inductive
	Constructions}\label{choice-principles-and-infinitary-inductive-constructions}

The use of WISC situates QW-type semantics within a broader landscape of
constructive set and type theories that admit weak choice principles
\cite{aczel1978-type-theoretic-interpretation,palmgren2002-developments-cst}.
While WISC is strictly weaker than full choice and is validated in many
predicative models, it is not derivable in intensional type theory and
its computational interpretation remains unclear. Understanding when
such principles are genuinely required is therefore an important
foundational question.

Infinitary inductive definitions provide natural test cases for this
analysis. Examples include countably branching trees modulo extensional
equality and type-theoretic presentations of ordinals, where equality is
defined by mutual inclusion of subtrees
\cite{coquand2010-constructively-finite}. In such cases, equations may
require recursive comparison at arbitrarily large depths, making it
difficult to bound the data needed to construct algebra structures at
colimit stages.

This paper investigates a contrasting infinitary example---mobiles, i.e.
well-founded trees quotiented by permutation of immediate subtrees---and
shows that, despite their infinitary branching, their defining equations
are sufficiently local to admit a fully constructive QW-type
construction without WISC.

\bibliography{master}

\end{document}
