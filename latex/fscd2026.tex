\PassOptionsToPackage{unicode}{hyperref}
\PassOptionsToPackage{hyphens}{url}
\documentclass[a4paper,UKenglish,cleveref, autoref, thm-restate]{lipics-v2021}
\usepackage{mathpartir}
\usepackage{amsmath}
\usepackage{mathtools}
\bibliographystyle{plainurl}

% Define missing control sequences
\providecommand{\coloneqq}{\mathrel{\mathop:}=}
\providecommand{\sym}{\sim}
\DeclareMathOperator*{\colim}{colim}

\providecommand{\tightlist}{%
\setlength{\itemsep}{0pt}\setlength{\parskip}{0pt}}

\title{Constructing QITs from Quotients}

\author{Thorsten Altenkirch}{University of Nottingham, UK}{thorsten.altenkirch@nottingham.ac.uk}{https://orcid.org/0000-0002-6582-5025}{}
\author{Christina O'Donnell}{University of Nottingham, UK}{psxco3@nottingham.ac.uk}{}{Supported by the Engineering and Physical Sciences Research Council (EPSRC) Doctoral Landscape Award [Grant Ref: EP/Z534948/1].}

\authorrunning{T. Altenkirch and C. D. O'Donnell}
\Copyright{Thorsten Altenkirch and Christina O'Donnell}

\ccsdesc[500]{Theory of computation~Type theory}
\ccsdesc[300]{Theory of computation~Constructive mathematics}
\ccsdesc[300]{Theory of computation~Logic and verification}

\keywords{Quotient Inductive Types, W-Types, Homotopy Type Theory, Agda, Constructive Mathematics}

% Optional: Link to the ArXiv version or repository if applicable
%\relatedversion{Full version hosted on arXiv: \url{https://arxiv.org/abs/xxxx.xxxxx}}

\supplement{Source code available at \url{https://github.com/cdo256/agda-qit}}

\begin{document}

\maketitle
\begin{abstract}
	Quotient Inductive Types (QITs) provide a powerful mechanism for defining
	datatypes equipped with identifications, and play an important role in
	constructive type theory and related semantic models. However, their general
	construction raises foundational difficulties: Shulman and Lumsdaine showed that
	certain QITs, including standard presentations of the countable ordinals, cannot
	be constructed using quotient types alone. Subsequent work by Fiore, Pitts, and
	Steenkamp introduced Quotient–W types (QW types), showing that a broad class of
	QITs admit initial algebra semantics assuming the Weak Initial Set of Covers
	(WISC) principle.

	In this work, we revisit the role of WISC in the construction of QW types. We
	focus on a minimal but non-trivial infinitary example: mobiles, defined as
	well-founded trees quotiented by permutation at each node. Mobiles have
	previously been considered to require choice principles in order to establish
	initiality. By formalising this example in Agda and analysing the
	Fiore–Pitts–Steenkamp construction in detail, we show that the use of WISC in
	this case can in fact be eliminated. We give a proof that infinite mobiles are
	cocontinuous, invoking Fiore et al. 2022 to show
	that mobiles admit an initial algebra, without appealing to WISC or any other choice
	principle.

	Our key observation is that the need for WISC arises when equations permit
	unbounded descent into inductive structure, as in extensional ordinal
	constructions. In contrast, the equations defining mobiles preserve tree depth
	and never require arbitrarily deep inspection of subtrees. Exploiting this
	boundedness allows us to establish the required cocontinuity argument
	constructively.

	This result provides evidence that some infinitary QITs previously thought to
	require choice are in fact constructible in a choice-free setting. We
	conclude by outlining a prospective syntactic criterion, based on bounding the
	depth of equations, that may characterise a larger class of constructible QITs.
\end{abstract}

%{
%	\setcounter{tocdepth}{3}
%	\tableofcontents
%}
\section{Introduction}\label{introduction}

Quotient inductive types (QITs) extend ordinary inductive types with
equational identifications and provide a uniform way to define datatypes
equipped with canonical equalities. They play an important role in
formalised mathematics in type theory, where they are used to present
structures such as Cauchy reals, partiality monads, and various
constructions of ordinals. Despite their usefulness, QITs are known to
pose subtle foundational and semantic difficulties, especially when
their constructors involve infinitary branching or when their path
constructors express extensionality principles.

Fiore, Pitts, and Steenkamp (FPS) introduced a general semantic
framework for a large class of QITs, called quotient--W types (QW
types). Their construction proceeds by building a size-indexed diagram
of stagewise approximations and taking a colimit, yielding an initial
algebra that supports the expected elimination principle. However, their
proof that constructor formation commutes with this colimit relies on
the Weak Initial Set of Covers principle (WISC), a weak choice principle
that is not derivable in standard predicative type theory. While WISC is
constructively acceptable in many settings, its use raises a natural
question: which features of a QIT genuinely require such principles, and
which arise from the generality of the FPS argument?

In this paper we investigate this question through a concrete infinitary
example: \textbf{mobiles}, i.e.~well-founded trees with countably many
children at each node, quotiented by permutation of subtrees. Mobiles
form a conceptually simple QIT with one infinitary constructor and one
path constructor expressing that the order of children is irrelevant.
Mobiles form a conceptually simple QIT with one infinitary constructor
and one path constructor expressing that the order of children is
irrelevant. In earlier work of Fiore, Pitts, and Steenkamp, this example
is presented as a representative infinitary quotient inductive type for
which their general construction relies on WISC.
\cite{fiore2020-qw-types}

Our main result is that mobiles admit an initial algebra semantics
\textbf{without using WISC}. We achieve this by refining the FPS
size-indexed construction. We observe that the reliance on WISC in the
FPS proof is tied to equations that require recursive inspection at
unbounded depth, as in extensional definitions of ordinals. In contrast,
the permutation equations for mobiles are \emph{depth-bounded}: they act
only at a single tree level and do not require comparing subtrees
recursively. We formalise a corresponding locality condition on QIT
signatures and show that, under this condition, the key cocontinuity
step in the FPS construction can be proved by well-founded recursion on
the size index alone.

Concretely, we define mobiles as a quotient of well-founded
\(\mathbb{N}\)-branching trees by levelwise permutation and construct
them as the colimit of a transfinite sequence of stagewise quotients. We
show that constructor formation preserves this colimit constructively:

\[
	\mathrm{colim}_\alpha F(D_\alpha) \cong F\Bigl(\mathrm{colim}_\alpha D_\alpha\Bigr)
\]

and derive the expected initial algebra property. The depth-preservation
property for mobiles and the resulting constructive cocontinuity proof have been
formalised in Agda. We rely on the general FPS initiality theorem
\cite{fiore2022-quotients-inductive-types} to obtain the elimination principle.

This identifies a new subclass of infinitary QITs that can be
constructed purely from quotient types, without additional choice
principles. It also clarifies the role of WISC in the FPS framework:
choice is needed when equations enforce global, unbounded recursive
comparison, but not when they are local to bounded depth.

\subsection{Contributions}\label{contributions}

\begin{itemize}
	\item
	      \textbf{Constructive construction of an infinitary QIT.} We give a
	      construction of the quotient inductive type of infinitary mobiles
	      (well-founded countably branching trees modulo permutation) that does
	      not rely on WISC or any other choice principle.
	\item
	      \textbf{Refinement of the Fiore--Pitts--Steenkamp framework.} We
	      analyse the size-indexed QW-type construction of Fiore, Pitts, and
	      Steenkamp and isolate the precise step where WISC is used, namely the
	      cocontinuity argument showing that constructor formation commutes with
	      the transfinite colimit of stagewise quotients.
	\item
	      \textbf{Depth-bounded (local) equations as a constructivity
		      criterion.} We identify a structural condition on QIT signatures,
	      formulated as depth-boundedness of equations, under which the
	      cocontinuity step can be proved constructively by well-founded
	      recursion on the size index.
	\item
	      \textbf{Application to mobiles.} We show that the permutation
	      equations for mobiles satisfy the depth-boundedness condition, and
	      therefore that the FPS construction yields an initial algebra for
	      mobiles without invoking WISC.
	\item
	      \textbf{Mechanised development.} The construction and main
	      metatheoretic arguments have been formalised in Agda, providing
	      machine-checked evidence for the correctness of the development.
	\item
	      \textbf{Conceptual separation between local and global equations.} Our
	      analysis explains the contrast between mobiles and extensional ordinal
	      constructions: the former involve only local (levelwise) equations,
	      while the latter require unbounded recursive comparison, clarifying
	      why choice principles arise in one case but not the other.
\end{itemize}

\section{Background}\label{background}

\subsection{Quotient Inductive Types}
\label{quotient-inductive-types}

Inductive types in dependent type theory are characterised by initial
algebra semantics for strictly positive endofunctors and admit
introduction, elimination, and computation rules justified by
well-founded recursion
\cite{martinlof1984-intuitionistic,dybjer1994-inductive-families}.
Categorically, such types correspond to initial algebras of polynomial
functors, whose structure can be described concretely using containers
\cite{abbott2005-containers} and, more generally, dependent polynomial
functors \cite{gambino2004-wellfounded-trees-polynomial}. These
semantics extend to constructive set-theoretic settings via well-founded
trees
\cite{moerdijk2000-wellfounded-trees,aczel1978-type-theoretic-interpretation,palmgren2002-developments-cst}.

Quotient inductive types (QITs) extend ordinary inductive types by
including path constructors that impose equations between generated
elements. They are typically considered in set-truncated settings, where
all higher paths are forced to be trivial
\cite{awodey2012-type-theory-homotopy,kraus2015-truncation-levels}. QITs
have been used to give intrinsic presentations of structures such as
partiality monads \cite{altenkirch2017-partiality-monad} and appear as a
special case of the more general quotient inductive-inductive types
(QIITs) \cite{altenkirch2018-quotient-inductive-inductive}.

Despite their expressive power, QITs do not in general admit a simple
initial algebra semantics. Lumsdaine and Shulman showed that certain
higher inductive types cannot be constructed in standard model
categories without additional assumptions
\cite{lumsdaine2020-semantics-higher-inductive}. Even in the
set-truncated case, equations that express extensionality over
infinitary data can obstruct straightforward constructions.

\subsubsection{QW Types and Size-Indexed
	Constructions}\label{qw-types-and-size-indexed-constructions}

Fiore, Pitts, and Steenkamp introduced \emph{quotient--W types} (QW
types) as a general semantic framework for a large class of QITs
\cite{fiore2020-qw-types,fiore2022-quotients-inductive-types,steenkamp2021-phd-thesis}.
A QW signature consists of a polynomial branching structure together
with a family of equations between terms. From such a signature they
construct a diagram of stagewise approximations indexed by a
well-founded size structure. At each stage, one forms a quotient of
well-founded trees whose subtrees come from earlier stages. The desired
QIT is obtained as the colimit of this diagram.

The key technical step in their development is a \emph{cocontinuity}
result: the polynomial functor determined by the signature must preserve
the size-indexed colimit. This ensures that the colimit carries an
algebra structure and satisfies the expected initiality property. In the
general case, the proof of cocontinuity relies on the Weak Initial Set
of Covers (WISC) axiom \cite{vandenberg2012-wisc-axiom}, a weak choice
principle that guarantees the existence of sufficiently small families
of covers. Intuitively, WISC is used to bound the branching behaviour of
families of subtrees so that constructor formation can be exchanged with
the colimit.

Related ideas appear in Pitts and Steenkamp's work on inflationary
iteration, where initial algebras for certain infinitary constructions
are obtained via transfinite iteration under boundedness assumptions
\cite{pitts2021-inflationary-iteration}. In both settings, well-founded
size indices play a role analogous to ordinal stages in classical
constructions of inductive definitions.

\subsubsection{Choice Principles and Infinitary Inductive
	Constructions}\label{choice-principles-and-infinitary-inductive-constructions}

The use of WISC situates QW-type semantics within a broader landscape of
constructive set and type theories that admit weak choice principles
\cite{aczel1978-type-theoretic-interpretation,palmgren2002-developments-cst}.
While WISC is strictly weaker than full choice and is validated in many
predicative models, it is not derivable in intensional type theory and
its computational interpretation remains unclear. Understanding when
such principles are genuinely required is therefore an important
foundational question.

Infinitary inductive definitions provide natural test cases for this
analysis. Examples include countably branching trees modulo extensional
equality and type-theoretic presentations of ordinals, where equality is
defined by mutual inclusion of subtrees
\cite{coquand2010-constructively-finite}. In such cases, equations may
require recursive comparison at arbitrarily large depths, making it
difficult to bound the data needed to construct algebra structures at
colimit stages.

This paper investigates a contrasting infinitary example---mobiles, i.e.
well-founded trees quotiented by permutation of immediate subtrees---and
shows that, despite their infinitary branching, their defining equations
are sufficiently local to admit a fully constructive QW-type
construction without WISC.

\section{Mobiles}

We study \emph{mobiles}, i.e.\ well-founded trees with infinitary branching,
quotiented by permutation of immediate subtrees. Fix a type $I$ of
branching indices, which may be infinite (for example $I \equiv \mathbb{N}$).
Informally, mobiles are generated by a leaf, a node constructor taking an
$I$-indexed family of subtrees, and an equation identifying nodes whose
children differ only by a permutation of $I$.

\medskip

\noindent\textbf{Signature.}
We consider the following QIT signature:
\[
	\begin{aligned}
		 & \mathsf{leaf} : M                                                \\
		 & \mathsf{node} : (I \to M) \to M                                  \\
		 & \mathsf{perm} : (\varphi : \mathsf{Iso}(I,I))\,(f : I \to M) \to
		\mathsf{node}(f) = \mathsf{node}(f \circ \varphi).
	\end{aligned}
\]
Here $\mathsf{Iso}(I,I)$ denotes bijections on $I$. The path constructor
$\mathsf{perm}$ enforces that the order of children at each node is
irrelevant.

\medskip

\noindent\textbf{Underlying polynomial.}
Ignoring equations, the point constructors correspond to the polynomial
endofunctor
\[
	F(X) \;\coloneqq\; 1 + (I \to X),
\]
whose initial algebra is the type of well-founded $I$-branching trees.
Mobiles arise by quotienting this W-type by the congruence generated by
the permutation equations.

\medskip

\noindent\textbf{Locality of equations.}
A key feature of this signature is that the equations act only at a
single tree level. The path constructor $\mathsf{perm}$ compares two
terms of the form $\mathsf{node}(f)$ and $\mathsf{node}(f\circ\varphi)$
without recursively inspecting the subtrees $f(i)$. This contrasts with
extensional equality principles for ordinals, where equality requires
mutual recursive comparison of subtrees at unbounded depth.

This locality property will be formalised in the next section as
\emph{depth-boundedness} of equations and will be used to show that the
QW-type construction for mobiles does not require WISC.

\bigskip

\section{Local QITs via Depth Preservation}
\label{sec:local-qits}

We now isolate a structural condition on QW signatures under which the
Fiore--Pitts--Steenkamp (FPS) size-indexed construction can be carried out
constructively. The key idea is that, for certain signatures, equality
proofs never relate terms of genuinely different depth. This allows the
size bounds in the FPS diagram to be reconstructed from the shapes of
terms themselves, eliminating the need for the WISC-based bounding
argument.

Unlike the FPS construction, which expands on free F-algebras as the we use the
depth of terms directly as their stage-wise expansion.

\subsection{Depth in the size-indexed diagram}

Let $T$ be the underlying W-type generated by the polynomial part of a
QW signature, before quotienting by equations. As in the FPS framework,
we assume a well-founded size index (ordinal-like) and a depth function
\[
	\mathrm{depth} : T \to \mathsf{Ord}
\]
assigning to each tree its rank. For ordinals $\alpha,\beta$ write
$\alpha \sym \beta$ for mutual boundedness.

At stage $\alpha$ of the FPS construction, elements of the approximant
$D_\alpha$ are represented by pairs
\[
	\hat s = (s, s \le \alpha)
\]
where $s:T$ and $s\le\alpha$ is a proof that $\mathrm{depth}~s \le \alpha$.

For each stage $\alpha$, we define a relation
\[
	\approx_\alpha \;\subseteq\; D_\alpha \times D_\alpha
\]
as the least relation satisfying the following clauses.

\begin{enumerate}
	\item \textbf{Congruence.}
	      If $f,g : B(a)\to D$ are families of subterms such that
	      $f(i) \approx_{\mu(i)} g(i)$ for all $i$, then
	      \[
		      \mathsf{sup}(a,f) \approx_\alpha \mathsf{sup}(a,g).
	      \]

	\item \textbf{Equation satisfaction.}
	      For every equation $e$ in the signature and every assignment of its
	      variables to terms bounded by $\alpha$, the instantiated left- and
	      right-hand sides are related by $\approx_\alpha$.

	\item \textbf{Equivalence closure.}
	      The relation $\approx_\alpha$ is reflexive, symmetric, and transitive.

	\item \textbf{Weakening.}
	      If $\alpha \le \beta$ and $\hat s \approx_\alpha \hat t$, then their
	      images in $D_\beta$ are related by $\approx_\beta$.
\end{enumerate}

\subsection{Depth-preserving equality}

Intuitively, a QIT is \emph{local} if its defining equations do not move
information between different depths of a term. We formalise this as
follows.

\begin{definition}[Depth-preserving QIT]
	A QW signature is \emph{depth-preserving} if for all stages $\alpha$ and
	all $\hat s,\hat t \in D_\alpha$,
	\[
		\hat s \approx_\alpha \hat t
		\quad\Rightarrow\quad
		\mathrm{depth}~s \sim \mathrm{depth}~t.
	\]
\end{definition}

Thus any stagewise equality proof implies that the underlying trees have
the same depth, up to mutual boundedness.

\subsection{Tight equality without explicit bounds}

Depth preservation allows us to replace bounded equality $\approx_\alpha$ by a
``tight'' equality on the underlying trees that reconstructs bounds from
their depths.

Define mutual boundedness on trees by
\[
	s \sim t \coloneqq \mathrm{depth}(s) \sim \mathrm{depth}(t)
\]

\begin{definition}[Tight equality]
	For $s,t:T$, define $s \approx^s t$ to consist of:
	\begin{enumerate}
		\item a proof that $s \sim t$, and
		\item a bounded equality proof
		      \[
			      (s,\_) \approx_{\mathrm{depth}(s)} (t,\_),
		      \]
		      where the bounds are reconstructed using the depth information.
	\end{enumerate}
\end{definition}

\begin{lemma}[Structural properties of $\approx^s$]
	\label{lem:tight-eq-structure}
	The relation $\approx^s$ on $T$ is an equivalence relation and is closed
	under constructors.
\end{lemma}

\begin{lemma}[Tightening]
	\label{lem:tightening}
	Suppose $\approx_\alpha$ is depth-preserving for all $\alpha$.
	If $\hat s \approx_\alpha \hat t$ in $D_\alpha$, then $s \approx^s t$.
\end{lemma}

\begin{proof}
	This proof is straightforward by induction on constructors of $\approx_\alpha$.
	Structural properties are given directly from Lemma
	\ref{lem:tight-eq-structure}. Weakening is irrelevant since the bound is
	ignored. The remaining case is if they are related by an equation. We know
	that $s \sim t$, so it remains to show that $s \approx_{\mathrm{depth}(s)} t$.
	This is done by showing the left and right sides of the equation can be
	tightened.
\end{proof}

The crucial point is that Lemma~\ref{lem:mobiles-depth-preserving}
supplies the required depth comparison needed to form a tight equality.

\subsection{Depth preservation and cocontinuity}
\label{sec:depth-cocontinuity}

Let $F$ be the polynomial functor determined by the point constructors of
the signature. In the mobile case one may take $F(X) \coloneqq 1 + (I \to X)$,
but the statements below are written for a general polynomial functor. Let
$D : \mathsf{Ord} \to \mathsf{Setoid}$ be the FPS stage diagram
$\alpha \mapsto D_\alpha$ with structure maps $D_\alpha \to D_\beta$ for
$\alpha \le \beta$, and write
\[
	Q \;\coloneqq\; \mathrm{colim}_\alpha D_\alpha
	\qquad\text{and}\qquad
	L \;\coloneqq\; \mathrm{colim}_\alpha (F \circ D)_\alpha.
\]
Cocontinuity of $F$ with respect to $D$ is the statement that the canonical map
\[
	\mathrm{colim}_\alpha F(D_\alpha) \longrightarrow F\Bigl(\mathrm{colim}_\alpha D_\alpha\Bigr)
\]
is an isomorphism. In \cite{fiore2022-quotients-inductive-types} this is the
place where WISC is used, in order to bound families of subtrees uniformly and
thereby construct an inverse map.

In contrast, for depth-preserving signatures we can build the inverse map
constructively, because the necessary bounds can be reconstructed from term
depths rather than chosen globally.

\subsubsection{The forward map.}
There is a canonical comparison map $\varphi : L \to F(Q)$ induced by the
colimit universal property. Concretely, an element of $L$ is represented by a
pair $(\alpha, x)$ with $x \in F(D_\alpha)$, and we define $\varphi$ by pushing
all $D_\alpha$-components into $Q$:
\[
	\varphi(\alpha, x) \;\coloneqq\; F(\iota_\alpha)(x),
\]
where $\iota_\alpha : D_\alpha \to Q$ is the colimit injection.

In the mobile case, writing $x$ as either $\mathsf{inl}(\star)$ or
$\mathsf{inr}(f)$ with $f:I\to D_\alpha$, we have
\[
	\varphi(\alpha,\mathsf{inl}(\star)) = \mathsf{inl}(\star),
	\qquad
	\varphi(\alpha,\mathsf{inr}(f)) = \mathsf{inr}(\lambda i.\ \iota_\alpha(f(i))).
\]

\subsubsection{The inverse map and the role of depth preservation.}
To construct $\psi : F(Q) \to L$ one must, given $f:I\to Q$, choose a stage at
which all the children of $f$ can be represented simultaneously. In the general
FPS argument, producing such a uniform bound is where WISC enters.

For depth-preserving signatures we avoid this by choosing a stage canonically
from the term data. Informally, for each $i:I$ pick a representative
\[
	f(i) = (\alpha_i, \hat t_i) \in Q,
\]
and take the stage $\alpha \coloneqq \sup_i \alpha_i$ (or any chosen join
operation on sizes). Then define
\[
	\psi(\mathsf{inr}(f)) \;\coloneqq\; \bigl(\alpha,\,\mathsf{inr}(g)\bigr)
	\quad\text{where}\quad
	g(i) \;\coloneqq\; \mathsf{weaken}_{\alpha_i\le\alpha}(\hat t_i) \in D_\alpha.
\]
The subtle point is well-definedness: $f(i)$ has many representatives, and
changing representatives must not change the resulting element of the colimit
$L$.

This is precisely where depth preservation is used. The ``tightening'' lemma
(Lemma~\ref{lem:tightening}) and its colimit analogue allow us to replace any
colimit equality proof by a tight equality proof whose bound is determined by
$\mathrm{depth}$ alone. As a result, transporting along stage inclusions becomes
inessential in equality proofs (the step case is trivial), and $\psi$ becomes
well-defined without any choice of covers.

\subsubsection{Congruence of $\varphi$ and $\psi$.}
To complete the construction one proves:
\begin{enumerate}
	\item $\varphi$ respects the colimit equality on $L$ (routine, by colimit
	      induction and stagewise congruence).
	\item $\psi$ respects the setoid equality on $F(Q)$.
	      In the mobile case, this reduces to showing that if
	      $f(i) =_{Q} g(i)$ for all $i$, then the stagewise families constructed for
	      $\psi(\mathsf{inr}(f))$ and $\psi(\mathsf{inr}(g))$ are equal in the
	      colimit. Here the key step is: from $f(i)=_Q g(i)$ obtain a tight equality
	      between underlying trees, then weaken it to the chosen join stage, and
	      finally apply congruence of the $F$-constructor.
\end{enumerate}

\subsubsection{Inverse laws.}
Finally, one proves the left and right inverse laws:
\[
	\varphi \circ \psi = \mathrm{id}_{F(Q)}
	\qquad\text{and}\qquad
	\psi \circ \varphi = \mathrm{id}_{L}.
\]
For mobiles, the first is immediate by inspecting constructors: $\mathsf{leaf}$
maps to $\mathsf{leaf}$ and $\mathsf{node}$ maps by pointwise colimit injection
followed by a single colimit step.

For the second, one shows that every representative $(\alpha,x)$ of $L$ is
equal, in the colimit, to the representative constructed by $\psi(\varphi(\alpha,x))$.
In the node case, this reduces to a single use of a canonical inequality
$\alpha \le \sup_{i:I}\alpha$ (or its analogue) witnessing that each child lies
below the joined stage.

\begin{theorem}[Constructive cocontinuity from depth preservation]
	\label{thm:constructive-cocontinuity}
	If the stagewise equality relations $\approx_\alpha$ are depth-preserving, then
	the comparison map $\varphi : \mathrm{colim}_\alpha F(D_\alpha) \to F(\mathrm{colim}_\alpha D_\alpha)$
	is an isomorphism. Equivalently, $F$ preserves the colimit of the FPS diagram
	constructively (without using WISC).
\end{theorem}

\begin{proof}[Proof sketch]
	Define $\varphi$ as above. Define $\psi$ by choosing the join stage of the
	indices of the children and weakening each child into that stage. Prove that
	$\psi$ is well-defined and respects equality by using tightening:
	any equality in $Q$ can be witnessed at a stage determined by
	$\mathrm{depth}$, so the step/weakening cases in the colimit equality become
	inessential. Finally prove the inverse laws by constructor case analysis and
	the colimit step equations.
\end{proof}

\begin{corollary}[Initial algebra without WISC]
	\label{cor:initial-algebra-depth}
	For a depth-preserving QW signature, the FPS colimit $Q$ carries an $F$-algebra
	structure satisfying the equations, and this algebra is initial, without
	assuming WISC.
\end{corollary}

\section{Constructing mobiles without WISC}

\begin{lemma}[Mobiles are depth-preserving]
	\label{lem:mobiles-depth-preserving}
	The QIT of mobiles is depth-preserving.
\end{lemma}

\begin{proof}[Proof]
	The only structural generating equation is permutation at a node:
	$\mathsf{node}(f) = \mathsf{node}(f \circ \varphi)$. This equation
	preserves the constructor shape, simply permuting the variables at depth 1.
	All children are preserved, so by quasi-extensionality, depth is preserved.
	Congruence and the equivalence rules for equality also
	preserve depth. Hence any derivation of $\approx_\alpha$ relates trees of the
	same depth.
\end{proof}

\bibliography{master}

\end{document}
