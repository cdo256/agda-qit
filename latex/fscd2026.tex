\PassOptionsToPackage{unicode}{hyperref}
\PassOptionsToPackage{hyphens}{url}
\documentclass[a4paper,UKenglish,cleveref, autoref, thm-restate]{lipics-v2021}
\usepackage{mathpartir}
\usepackage{amsmath}
\usepackage{mathtools}
\bibliographystyle{plainurl}

% Define missing control sequences
\providecommand{\coloneqq}{\mathrel{\mathop:}=}
\providecommand{\sym}{\sim}
\DeclareMathOperator*{\colim}{colim}

\providecommand{\tightlist}{%
\setlength{\itemsep}{0pt}\setlength{\parskip}{0pt}}

\title{Depth-Preserving QW Types are Constructive}

\author{Thorsten Altenkirch}{University of Nottingham, UK}{thorsten.altenkirch@nottingham.ac.uk}{https://orcid.org/0000-0002-6582-5025}{}
\author{Christina O'Donnell}{University of Nottingham, UK}{psxco3@nottingham.ac.uk}{}{Supported by the Engineering and Physical Sciences Research Council (EPSRC) Doctoral Landscape Award [Grant Ref: EP/Z534948/1].}

\authorrunning{T. Altenkirch and C. D. O'Donnell}
\Copyright{Thorsten Altenkirch and Christina O'Donnell}

\ccsdesc[500]{Theory of computation~Type theory}
\ccsdesc[300]{Theory of computation~Constructive mathematics}
\ccsdesc[300]{Theory of computation~Logic and verification}

\keywords{Quotient Inductive Types, W-Types, Homotopy Type Theory, Agda, Constructive Mathematics}

\supplement{Source code available at \url{https://github.com/cdo256/agda-qit}}

\begin{document}

\maketitle
\begin{abstract}
	Quotient Inductive Types (QITs) provide a powerful mechanism for defining
	datatypes equipped with identifications. However, their general
	construction raises foundational difficulties: certain QITs cannot
	be constructed using quotient types alone. Subsequent work by Fiore, Pitts, and
	Steenkamp introduced Quotient--W types (QW types), showing that a broad class of
	QITs admit initial algebra semantics assuming the Weak Initial Set of Covers
	(WISC) principle.

	We identify a structural condition on QW signatures, depth preservation, and
	prove that all depth-preserving QW types admit constructive initial algebra
	semantics, eliminating the use of WISC in the Fiore--Pitts--Steenkamp
	construction. As an application, we obtain a fully constructive construction
	of infinitary mobiles. Our key observation is that the need for WISC arises
	when equations permit unbounded descent into inductive structure, as in
	extensional ordinal constructions. In contrast, depth-preserving equations
	ensure that stagewise equalities do not change term depth. Exploiting this
	boundedness allows us to establish the required cocontinuity argument
	constructively.
\end{abstract}

\section{Introduction}\label{introduction}

Quotient inductive types (QITs) extend ordinary inductive types with
equational identifications and provide a uniform way to define datatypes
equipped with canonical equalities. Despite their usefulness, QITs are known to
pose subtle foundational and semantic difficulties, especially when
their constructors involve infinitary branching or when their path
constructors express extensionality principles.

Fiore, Pitts, and Steenkamp (FPS) introduced a general semantic
framework for a large class of QITs, called quotient--W types (QW
types). Their proof that constructor formation commutes with the size-indexed
colimit relies on the Weak Initial Set of Covers principle (WISC), a weak choice
principle that is not derivable in standard predicative type theory.

In this paper, we identify a structural class of QW types whose semantics is
fully constructive. We introduce the notion of \emph{depth preservation} and
show that for signatures satisfying this property, the FPS construction
succeeds without WISC. We investigate this through the case study of
\textbf{mobiles}, i.e.~well-founded trees with countably many children at each
node, quotiented by permutation of subtrees. While previously thought to
require choice, we show that mobiles are depth-preserving and thus admit a
constructive initial algebra semantics.

\subsection{Contributions}\label{contributions}

\begin{itemize}
	\item \textbf{Depth preservation as a structural property.} We introduce
	      depth preservation for QW signatures, ensuring that stagewise equalities do
	      not change term depth.
	\item \textbf{General constructive cocontinuity.} We prove that all
	      depth-preserving QW types are constructively cocontinuous, i.e., the FPS
	      colimit commutes with the signature functor without WISC.
	\item \textbf{Mechanisation.} We mechanise this general result in Agda and
	      verify the hypotheses for infinitary mobiles.
	\item \textbf{Constructive construction of mobiles.} As a corollary, we
	      obtain a fully constructive initial algebra semantics for infinitary
	      mobiles, providing evidence that this class of QITs does not require choice.
\end{itemize}

\section{Background}\label{background}

\subsection{Quotient Inductive Types}
\label{quotient-inductive-types}

Inductive types in dependent type theory are characterised by initial
algebra semantics for strictly positive endofunctors. Quotient inductive types
(QITs) extend these by including path constructors that impose equations
between generated elements. Even in the set-truncated case, equations that
express extensionality over infinitary data can obstruct straightforward
constructions.

\subsubsection{QW Types and Size-Indexed
	Constructions}\label{qw-types-and-size-indexed-constructions}

Fiore, Pitts, and Steenkamp introduced \emph{quotient--W types} (QW
types) as a general semantic framework for QITs. They construct a diagram of
stagewise approximations indexed by a well-founded size structure. The desired
QIT is the colimit of this diagram. The key technical step is a
\emph{cocontinuity} result: the polynomial functor must preserve the colimit.
In the general case, this proof relies on WISC to bound the branching
behaviour of families of subtrees.

\section{Mobiles}

We study \emph{mobiles}, well-founded trees with infinitary branching
indices $I$, quotiented by permutation of immediate subtrees.

\medskip

\noindent\textbf{Signature.}
\[
	\begin{aligned}
		 & \mathsf{leaf} : M                                                \\
		 & \mathsf{node} : (I \to M) \to M                                  \\
		 & \mathsf{perm} : (\varphi : \mathsf{Iso}(I,I))\,(f : I \to M) \to
		\mathsf{node}(f) = \mathsf{node}(f \circ \varphi).
	\end{aligned}
\]

A key feature is that the equations act only at a single tree level. This
locality suggests that the size bounds needed for cocontinuity can be
reconstructed from the terms themselves.

\section{Depth-Preserving QW Types are Constructive}
\label{sec:local-qits}

We now isolate a structural condition on QW signatures under which the
FPS construction can be carried out constructively.

\subsection{Stage structure recap}

Let $T$ be the underlying W-type. We assume a well-founded size index and a
depth function $\mathrm{depth} : T \to \mathsf{Ord}$. At stage $\alpha$,
elements of $D_\alpha$ are pairs $\hat s = (s, s \le \alpha)$ where $s:T$.
The relation $\approx_\alpha \subseteq D_\alpha \times D_\alpha$ is defined
via congruence, equation satisfaction, equivalence closure, and weakening.

\subsection{Definition of depth preservation}

\begin{definition}[Depth-preserving QW signature]
	A QW signature is \emph{depth-preserving} if for all stages $\alpha$ and
	all $\hat s,\hat t \in D_\alpha$,
	$ \hat s \approx_\alpha \hat t \Rightarrow \mathrm{depth}~s \sim \mathrm{depth}~t. $
\end{definition}

\subsection{Tight equality construction}

We define a ``tight'' equality on underlying trees that reconstructs bounds.
For $s,t:T$, $s \approx^s t$ consists of a proof $s \sim t$ and a proof
$(s,\_) \approx_{\mathrm{depth}(s)} (t,\_)$.

\subsection{Colimit tightening lemma}

\begin{lemma}[Tightening]
	\label{lem:tightening}
	Suppose $\approx_\alpha$ is depth-preserving for all $\alpha$.
	If $\hat s \approx_\alpha \hat t$ in $D_\alpha$, then $s \approx^s t$.
\end{lemma}

\subsection{General constructive cocontinuity theorem}

\begin{theorem}
	If a QW signature is depth-preserving, then the FPS colimit construction is
	cocontinuous without WISC.
\end{theorem}

\begin{proof}[Proof sketch]
	Construct the inverse map $\psi : F(Q) \to L$ by choosing the join stage
	of indices of children. Tightening ensures $\psi$ is well-defined
	without a choice of covers, as any equality in the colimit can be
	witnessed at a stage determined by $\mathrm{depth}$.
\end{proof}

\subsection{Corollary: initial algebra without WISC}

\begin{corollary}
	Depth-preserving QW types admit initial algebra semantics in MLTT with
	quotients alone.
\end{corollary}

In Section~\ref{sec:mobiles-app} we verify that mobiles satisfy depth preservation.

\section{Infinitary Mobiles as a Depth-Preserving QW Type}
\label{sec:mobiles-app}

\begin{lemma}[Mobiles are depth-preserving]
	\label{lem:mobiles-depth-preserving}
	The QIT of mobiles is depth-preserving.
\end{lemma}

\begin{proof}
	The permutation equation $\mathsf{node}(f) = \mathsf{node}(f \circ \varphi)$
	simply permutes variables at depth 1. Since all children are preserved, any
	derivation of $\approx_\alpha$ relates trees of the same depth.
\end{proof}

By instantiating Theorem~\ref{thm:constructive-cocontinuity}, we conclude that
infinitary mobiles admit a fully constructive initial algebra semantics.

\section{Formalisation Statement}

We mechanise the general theorem that depth-preserving QW types are
cocontinuous, and instantiate it to mobiles. The development in Agda
verifies that the cocontinuity argument for depth-preserving signatures does
not rely on choice principles.

\section{Conclusion}

We show that depth-preserving QW types form a class of infinitary quotient
inductive types that admit constructive semantics. This isolates a precise
source of non-constructivity in the FPS construction—unbounded recursive
comparison—and suggests a syntactic criterion for identifying QITs that do
not require choice.

\bibliography{master}

\end{document}
