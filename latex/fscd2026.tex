\PassOptionsToPackage{unicode}{hyperref}
\PassOptionsToPackage{hyphens}{url}
\documentclass[a4paper,UKenglish,cleveref, autoref, thm-restate]{lipics-v2021}
\usepackage{mathpartir}
\usepackage{amsmath}
\usepackage{mathtools}
\bibliographystyle{plainurl}

\title{Depth-Preserving QW Types are Constructive}

\author{Thorsten Altenkirch}{University of Nottingham, UK}
    {thorsten.altenkirch@nottingham.ac.uk}{https://orcid.org/0000-0002-6582-5025}{}
\author{Christina O'Donnell}{University of Nottingham, UK}
    {psxco3@nottingham.ac.uk}{}{Supported by the Engineering and Physical Sciences Research Council (EPSRC) Doctoral Landscape Award [Grant Ref: EP/Z534948/1].}

\authorrunning{T. Altenkirch and C. D. O'Donnell}
\Copyright{Thorsten Altenkirch and Christina O'Donnell}

\ccsdesc[500]{Theory of computation~Type theory}
\ccsdesc[300]{Theory of computation~Constructive mathematics}
\ccsdesc[300]{Theory of computation~Logic and verification}

\keywords{Quotient Inductive Types, W-Types, Homotopy Type Theory, Agda, Constructive Mathematics}

% Optional: Link to the ArXiv version or repository if applicable
%\relatedversion{Full version hosted on arXiv: \url{https://arxiv.org/abs/xxxx.xxxxx}}

\supplement{Source code available at \url{https://github.com/cdo256/agda-qit}}

\begin{document}

\maketitle

\begin{abstract}
	Quotient Inductive Types (QITs) provide a powerful mechanism for defining
	datatypes equipped with identifications. However, their general construction
	raises foundational difficulties: certain QITs cannot be constructed using
	quotient types alone. Subsequent work by Fiore, Pitts, and Steenkamp
	introduced Quotient--W types (QW types), showing that a broad class of QITs
	admit initial algebra semantics assuming the Weak Initial Set of Covers
	(WISC) principle.

	We identify a structural condition on QW signatures, depth preservation, and
	prove that all depth-preserving QW types admit constructive initial algebra
	semantics, eliminating the use of WISC in the Fiore--Pitts--Steenkamp
	construction. As an application, we obtain a fully constructive construction
	of infinitary mobiles. Our key observation is that the need for WISC arises
	when equations permit unbounded descent into inductive structure, as in
	extensional ordinal constructions. In contrast, depth-preserving equations
	ensure that stagewise equalities do not change term depth. Exploiting this
	boundedness allows us to establish the required cocontinuity argument
	constructively.

	This result provides evidence that some infinitary QITs previously thought to
	require choice are in fact constructible in a choice-free setting. We
	conclude by outlining a prospective syntactic criterion, based on bounding the
	depth of equations, that may characterise a larger class of constructible QITs.
\end{abstract}

%{
%	\setcounter{tocdepth}{3}
%	\tableofcontents
%}
\section{Introduction}\label{introduction}

Quotient inductive types (QITs) extend ordinary inductive types with
equational identifications and provide a uniform way to define datatypes
equipped with canonical equalities. They play an important role in
formalised mathematics in type theory, where they are used to present
structures such as Cauchy reals, partiality monads, and various
constructions of ordinals. Despite their usefulness, QITs are known to
pose subtle foundational and semantic difficulties, especially when
their constructors involve infinitary branching or when their path
constructors express extensionality principles.

Fiore, Pitts, and Steenkamp (FPS) introduced a general semantic
framework for a large class of QITs, called quotient--W types (QW
types). Their construction proceeds by building a size-indexed diagram
of stagewise approximations and taking a colimit, yielding an initial
algebra that supports the expected elimination principle. However, their
proof that constructor formation commutes with this colimit relies on
the Weak Initial Set of Covers principle (WISC), a weak choice principle
that is not derivable in standard predicative type theory. While WISC is
constructively acceptable in many settings, its use raises a natural
question: which features of a QIT genuinely require such principles, and
which arise from the generality of the FPS argument?

In this paper we investigate this question through a concrete infinitary
example: \textbf{mobiles}, i.e.~well-founded trees with countably many
children at each node, quotiented by permutation of subtrees. Mobiles
form a conceptually simple QIT with one infinitary constructor and one
path constructor expressing that the order of children is irrelevant.
Mobiles form a conceptually simple QIT with one infinitary constructor
and one path constructor expressing that the order of children is
irrelevant. In earlier work of Fiore, Pitts, and Steenkamp, this example
is presented as a representative infinitary quotient inductive type for
which their general construction relies on WISC.
\cite{fiore2020-qw-types}

%TODO: Check which ordinals are difficult.
Our main result is that all depth-preserving QW types admit an initial algebra semantics
\textbf{without using WISC}. We achieve this by refining the FPS
size-indexed construction. We observe that the reliance on WISC in the
FPS proof is tied to equations that require recursive inspection at
unbounded depth, as in extensional definitions of ordinals. In contrast,
the permutation equations for mobiles are \emph{depth-preserving}: they act
only at a single tree level and do not require comparing subtrees
recursively. We formalise this depth-condition on QIT
signatures and show that, under this condition, the key cocontinuity
step in the FPS construction can be proved by well-founded recursion on
the size index alone.

Concretely, we define mobiles as a quotient of well-founded
\(\mathbb{N}\)-branching trees by levelwise permutation and construct
them as the colimit of a transfinite sequence of stagewise quotients. We
show that constructor formation preserves this colimit constructively:

\[
	\mathrm{colim}_\alpha F(D_\alpha) \cong F\Bigl(\mathrm{colim}_\alpha D_\alpha\Bigr)
\]

and derive the expected initial algebra property. The depth-preservation
property for mobiles and the resulting constructive cocontinuity proof have been
formalised in Agda. We rely on the general FPS initiality theorem
\cite{fiore2022-quotients-inductive-types} to obtain the elimination principle
and show it is unique.

This identifies a new subclass of infinitary QITs that can be
constructed purely from quotient types, without additional choice
principles. It also clarifies the role of WISC in the FPS framework:
choice is needed when equations enforce global, unbounded recursive
comparison, but not when they are local to bounded depth.

\subsection{Contributions}\label{contributions}

\begin{itemize}
	\item \textbf{Depth preservation as a structural property.}
	      We introduce the notion of depth preservation for QW
	      signatures, ensuring that stagewise equalities do
	      not change term depth.
	\item \textbf{General constructive cocontinuity.}
	      We prove that all depth-preserving QW types are constructively cocontinuous,
	      i.e., the FPS colimit commutes with the signature functor without WISC.
	\item \textbf{Formalisation.}
	      We formalise this general result in Agda and verify the hypotheses for
	      infinitary mobiles.
	\item \textbf{Constructive construction of mobiles.}
	      As a corollary, we obtain a fully constructive initial algebra semantics for
	      infinitary mobiles, providing evidence that this class of QITs does not
	      require choice.
\end{itemize}

\section{Background}\label{background}

\subsection{Quotient Inductive Types}
\label{quotient-inductive-types}

Inductive types in dependent type theory are characterised by initial
algebra semantics for strictly positive endofunctors and admit
introduction, elimination, and computation rules justified by
well-founded recursion
\cite{martinlof1984-intuitionistic,dybjer1994-inductive-families}.
Categorically, such types correspond to initial algebras of polynomial
functors, whose structure can be described concretely using containers
\cite{abbott2005-containers} and, more generally, dependent polynomial
functors \cite{gambino2004-wellfounded-trees-polynomial}. These
semantics extend to constructive set-theoretic settings via well-founded
trees
\cite{moerdijk2000-wellfounded-trees,aczel1978-type-theoretic-interpretation,palmgren2002-developments-cst}.

Quotient inductive types (QITs) extend ordinary inductive types by
including path constructors that impose equations between generated
elements. They are typically considered in set-truncated settings, where
all higher paths are forced to be trivial
\cite{awodey2012-type-theory-homotopy,kraus2015-truncation-levels}. QITs
have been used to give intrinsic presentations of structures such as
partiality monads \cite{altenkirch2017-partiality-monad} and appear as a
special case of the more general quotient inductive-inductive types
(QIITs) \cite{altenkirch2018-quotient-inductive-inductive}.

Despite their expressive power, QITs do not in general admit a simple
initial algebra semantics. Lumsdaine and Shulman showed that certain
higher inductive types cannot be constructed in standard model
categories without additional assumptions
\cite{lumsdaine2020-semantics-higher-inductive}. Even in the
set-truncated case, equations that express extensionality over
infinitary data can obstruct straightforward constructions.

\subsubsection{QW Types and Size-Indexed
	Constructions}\label{qw-types-and-size-indexed-constructions}

Fiore, Pitts, and Steenkamp introduced \emph{quotient--W types} (QW
types) as a general semantic framework for a large class of QITs
\cite{fiore2020-qw-types,fiore2022-quotients-inductive-types,steenkamp2021-phd-thesis}.
A QW signature consists of a polynomial branching structure together
with a family of equations between terms. From such a signature they
construct a diagram of stagewise approximations indexed by a
well-founded size structure. At each stage, one forms a quotient of
well-founded trees whose subtrees come from earlier stages. The desired
QIT is obtained as the colimit of this diagram.

The key technical step in their development is a \emph{cocontinuity}
result: the polynomial functor determined by the signature must preserve
the size-indexed colimit. This ensures that the colimit carries an
algebra structure and satisfies the expected initiality property. In the
general case, the proof of cocontinuity relies on the Weak Initial Set
of Covers (WISC) axiom \cite{vandenberg2012-wisc-axiom}, a weak choice
principle that guarantees the existence of sufficiently small families
of covers. Intuitively, WISC is used to bound the branching behaviour of
families of subtrees so that constructor formation can be exchanged with
the colimit.

Related ideas appear in Pitts and Steenkamp's work on inflationary
iteration, where initial algebras for certain infinitary constructions
are obtained via transfinite iteration under boundedness assumptions
\cite{pitts2021-inflationary-iteration}. In both settings, well-founded
size indices play a role analogous to ordinal stages in classical
constructions of inductive definitions.

\subsubsection{Choice Principles and Infinitary Inductive Constructions}
\label{choice-principles-and-infinitary-inductive-constructions}

The use of WISC situates QW-type semantics within a broader landscape of
constructive set and type theories that admit weak choice principles
\cite{aczel1978-type-theoretic-interpretation,palmgren2002-developments-cst}.
While WISC is strictly weaker than full choice and is validated in many
predicative models, it is not derivable in intensional type theory and
its computational interpretation remains unclear. Understanding when
such principles are genuinely required is therefore an important
foundational question.

Infinitary inductive definitions provide natural test cases for this
analysis. Examples include countably branching trees modulo extensional
equality and type-theoretic presentations of ordinals, where equality is
defined by mutual inclusion of subtrees
\cite{coquand2010-constructively-finite}. In such cases, equations may
require recursive comparison at arbitrarily large depths, making it
difficult to bound the data needed to construct algebra structures at
colimit stages.

This paper investigates a contrasting infinitary example---mobiles, i.e.
well-founded trees quotiented by permutation of immediate subtrees---and
shows that, despite their infinitary branching, their defining equations
are sufficiently local to admit a fully constructive QW-type
construction without WISC.
Inductive types in dependent type theory are characterised by initial algebra
semantics for strictly positive endofunctors. Quotient inductive types (QITs)
extend these by including path constructors that impose equations between
generated elements. Even in the set-truncated case, equations that express
extensionality over infinitary data can obstruct straightforward constructions.

\subsubsection{QW Types and Size-Indexed Constructions}
\label{qw-types-and-size-indexed-constructions}

Fiore, Pitts, and Steenkamp introduced \emph{quotient--W types} (QW types) as a
general semantic framework for QITs. They construct a diagram of stagewise
approximations indexed by a well-founded size structure. The desired QIT is the
colimit of this diagram. The key technical step is a \emph{cocontinuity} result:
the polynomial functor must preserve the colimit. In the general case, this
proof relies on WISC to bound the branching behaviour of families of subtrees.

\section{Mobiles}

We study \emph{mobiles}, i.e.\ well-founded trees with infinitary branching,
quotiented by permutation of immediate subtrees. Fix a type $I$ of
branching indices, which may be infinite (for example $I \equiv \mathbb{N}$).
Informally, mobiles are generated by a leaf, a node constructor taking an
$I$-indexed family of subtrees, and an equation identifying nodes whose
children differ only by a permutation of $I$.

\medskip

\noindent\textbf{Signature.}
We consider the following QIT signature:
\[
	\begin{aligned}
		 & \mathsf{leaf} : M                                                \\
		 & \mathsf{node} : (I \to M) \to M                                  \\
		 & \mathsf{perm} : (\varphi : \mathsf{Iso}(I,I))\,(f : I \to M) \to
		\mathsf{node}(f) = \mathsf{node}(f \circ \varphi).
	\end{aligned}
\]
Here $\mathsf{Iso}(I,I)$ denotes bijections on $I$. The path constructor
$\mathsf{perm}$ enforces that the order of children at each node is
irrelevant.

\medskip

\noindent\textbf{Underlying polynomial.}
Ignoring equations, the point constructors correspond to the polynomial
endofunctor
\[
	F(X) \;\coloneqq\; 1 + (I \to X),
\]
whose initial algebra is the type of well-founded $I$-branching trees.
Mobiles arise by quotienting this W-type by the congruence generated by
the permutation equations.

\medskip

\noindent\textbf{Locality of equations.}
A key feature of this signature is that the equations act only at a
single tree level. The path constructor $\mathsf{perm}$ compares two
terms of the form $\mathsf{node}(f)$ and $\mathsf{node}(f\circ\varphi)$
without recursively inspecting the subtrees $f(i)$. This contrasts with
extensional equality principles for ordinals, where equality requires
mutual recursive comparison of subtrees at unbounded depth.

This locality property will be formalised in the next section as
\emph{depth-boundedness} of equations and will be used to show that the
QW-type construction for mobiles does not require WISC.
A key feature is that the equations act only at a single tree level. This
locality suggests that the size bounds needed for cocontinuity can be
reconstructed from the terms themselves.

\bigskip

\section{Depth-Preserving QW Types are Constructive}
\label{sec:local-qits}

We now isolate a structural condition on QW signatures under which the
Fiore--Pitts--Steenkamp (FPS) size-indexed construction can be carried out
constructively. The key idea is that, for certain signatures, equality
proofs never relate terms of genuinely different depth. This allows the
size bounds in the FPS diagram to be reconstructed from the shapes of
terms themselves, eliminating the need for the WISC-based bounding
argument.

Unlike the FPS construction, which expands on free F-algebras as the we use the
depth of terms directly as their stage-wise expansion.

\subsection{Depth in the size-indexed diagram}

Let $T$ be the underlying W-type generated by the polynomial part of a
QW signature, before quotienting by equations. As in the FPS framework,
we assume a well-founded size index (ordinal-like) and a depth function
\[
	\mathrm{depth} : T \to \mathsf{Ord}
\]
assigning to each tree its rank. For ordinals $\alpha,\beta$ write
$\alpha \sym \beta$ for mutual boundedness.

At stage $\alpha$ of the FPS construction, elements of the approximant
$D_\alpha$ are represented by pairs
\[
	\hat s = (s, s \le \alpha)
\]
where $s:T$ and $s\le\alpha$ is a proof that $\mathrm{depth}~s \le \alpha$.

For each stage $\alpha$, we define a relation
\[
	\approx_\alpha \;\subseteq\; D_\alpha \times D_\alpha
\]
as the least relation satisfying the following clauses.

\begin{enumerate}
	\item \textbf{Congruence.}
	      If $f,g : B(a)\to D$ are families of subterms such that
	      $f(i) \approx_{\mu(i)} g(i)$ for all $i$, then
	      \[
		      \mathsf{sup}(a,f) \approx_\alpha \mathsf{sup}(a,g).
	      \]

	\item \textbf{Equation satisfaction.}
	      For every equation $e$ in the signature and every assignment of its
	      variables to terms bounded by $\alpha$, the instantiated left- and
	      right-hand sides are related by $\approx_\alpha$.

	\item \textbf{Equivalence closure.}
	      The relation $\approx_\alpha$ is reflexive, symmetric, and transitive.

	\item \textbf{Weakening.}
	      If $\alpha \le \beta$ and $\hat s \approx_\alpha \hat t$, then their
	      images in $D_\beta$ are related by $\approx_\beta$.
\end{enumerate}

\subsection{Stage structure recap}

Intuitively, a QIT is \emph{local} if its defining equations do not move
information between different depths of a term. We formalise this as
follows.

\begin{definition}[Depth-preserving QIT]
	A QW signature is \emph{depth-preserving} if for all stages $\alpha$ and
	all $\hat s,\hat t \in D_\alpha$,
	\[
		\hat s \approx_\alpha \hat t
		\quad\Rightarrow\quad
		\mathrm{depth}~s \sim \mathrm{depth}~t.
	\]
Let $T$ be the underlying W-type. We assume a well-founded size index and a
depth function $\mathrm{depth} : T \to \mathsf{Ord}$. For each stage $\alpha$,
we define a relation $\approx_\alpha \;\subseteq\; D_\alpha \times D_\alpha$ as
the least relation satisfying congruence, equation satisfaction, equivalence
closure, and weakening. Unlike the general FPS case, we use the depth of terms
directly to bound their stagewise expansion.

\subsection{Definition of depth preservation}

\begin{definition}[Depth-preserving QW signature]
	A QW signature is \emph{depth-preserving} if for all stages $\alpha$ and all
	$\hat s,\hat t \in D_\alpha$, $\hat s \approx_\alpha \hat t \Rightarrow
		\mathrm{depth}~s \sim \mathrm{depth}~t.$
\end{definition}

\subsection{Tight equality construction}

Depth preservation allows us to replace bounded equality by a ``tight'' equality
on the underlying trees. Define mutual boundedness on trees by
$s \sim t \coloneqq \mathrm{depth}(s) \sim \mathrm{depth}(t)$.
\begin{definition}[Tight equality]
	For $s,t:T$, define $s \approx^s t$ to consist of:
	\begin{enumerate}
		\item a proof that $s \sim t$, and
		\item a bounded equality proof $(s,\_) \approx_{\mathrm{depth}(s)} (t,\_)$,
		      where the bounds are reconstructed using depth information.
	\end{enumerate}
\end{definition}

\subsection{Colimit tightening lemma}

\begin{lemma}[Tightening]
	\label{lem:tightening}
	Suppose $\approx_\alpha$ is depth-preserving for all $\alpha$.
	If $\hat s \approx_\alpha \hat t$ in $D_\alpha$, then $s \approx^s t$.
\end{lemma}

\subsection{General constructive cocontinuity theorem}

\begin{theorem}
	If a QW signature is depth-preserving, then the FPS colimit construction is
	cocontinuous without WISC.
\end{theorem}

\begin{proof}[Proof sketch]
	Let $Q \coloneqq \mathrm{colim}_\alpha D_\alpha$ and $L \coloneqq
		\mathrm{colim}_\alpha (F \circ D)_\alpha$. We construct the inverse map
	$\psi : F(Q) \to L$ by choosing the join stage of indices of children.
	Tightening ensures $\psi$ is well-defined without a choice of covers, as any
	equality in the colimit can be witnessed at a stage determined by
	$\mathrm{depth}$ alone. This eliminates the reliance on WISC for uniform
	bounds.
\end{proof}

\subsection{Corollary: initial algebra without WISC}

\begin{corollary}
	Depth-preserving QW types admit initial algebra semantics in MLTT with
	quotients alone.
\end{corollary}

\section{Infinitary mobiles as a depth-preserving QW type}
\label{sec:mobiles}

In this section we instantiate the general result of
Section~\ref{sec:local-qits}: \emph{depth-preserving QW signatures are
	constructively cocontinuous}. We do so for the QW signature of
\emph{mobiles}, i.e.\ $I$-branching trees quotiented by permutation of
children at each node.

\subsection{The mobile signature}

Fix a type $I$ of branch indices. Consider the container (polynomial)
signature with shapes
\[
	S \;\coloneqq\; \{\mathsf{leaf},\mathsf{node}\}
\]
and arities
\[
	P(\mathsf{leaf}) \coloneqq \varnothing,
	\qquad
	P(\mathsf{node}) \coloneqq I.
\]
The associated polynomial endofunctor is
\[
	F(X) \;\coloneqq\; \sum_{s:S} (P(s)\to X)
	\;\cong\;
	1 + (I \to X),
\]
where we identify $\mathsf{leaf}$ with the left summand and $\mathsf{node}$
with the right summand.

The equations $E$ are indexed by bijections on $I$:
\[
	E \coloneqq \mathsf{Iso}(I,I),
\]
and each $\pi \in E$ generates the equation schema
\[
	\mathsf{node}(x_i)_{i:I} \;=\; \mathsf{node}(x_{\pi(i)})_{i:I},
\]
i.e.\ node formation is invariant under permuting the $I$-indexed family
of children.

This is a QW signature in the sense of Fiore--Pitts--Steenkamp
\cite{fiore2022-quotients-inductive-types}.

\subsection{Depth preservation for mobiles}

Recall from Section~\ref{sec:local-qits} that a QW signature is
\emph{depth-preserving} if for all stages $\alpha$ and all bounded
representatives $\hat s,\hat t\in D_\alpha$,
\[
	\hat s \approx_\alpha \hat t
	\quad\Rightarrow\quad
	\mathrm{depth}(s) \sim \mathrm{depth}(t).
\]

\begin{lemma}[Mobiles are depth-preserving]
	\label{lem:mobiles-depth-preserving-sec5}
	The mobile signature is depth-preserving.
\end{lemma}

\begin{proof}[Proof sketch]
	By induction on the derivation of $\hat s \approx_\alpha \hat t$.
	The equivalence-closure rules preserve depth immediately. For congruence,
	depth preservation follows from the induction hypotheses on corresponding
	subtrees and the definition of depth for W-types as a supremum of child
	depths.

	For the equation rule, the two sides of the instantiated permutation
	equation are
	\[
		\mathsf{node}(f)
		\qquad\text{and}\qquad
		\mathsf{node}(f\circ \pi),
	\]
	which have the same outer constructor and the same multiset of
	children depths; hence their depths coincide. \qedhere
\end{proof}

\subsection{Constructive cocontinuity and existence of mobiles}

Let $D:\mathsf{Ord}\to\mathsf{Setoid}$ be the FPS stage diagram for the
mobile signature, and let $Q\coloneqq \mathrm{colim}_\alpha D_\alpha$ be
its colimit. By Theorem~\ref{thm:constructive-cocontinuity}
(Section~\ref{sec:local-qits}),
depth preservation implies cocontinuity of the polynomial functor $F$:
\[
	\mathrm{colim}_\alpha F(D_\alpha) \;\cong\; F\!\left(\mathrm{colim}_\alpha D_\alpha\right)
	\quad\text{constructively.}
\]

\begin{theorem}[Cocontinuity for mobiles without WISC]
	\label{thm:mobiles-cocontinuity}
	The mobile stage diagram is cocontinuous without assuming WISC.
\end{theorem}

\begin{proof}
	Immediate from Lemma~\ref{lem:mobiles-depth-preserving-sec5} and
	Theorem~\ref{thm:constructive-cocontinuity}. \qedhere
\end{proof}

Finally, we appeal to the general FPS initiality theorem, which states
that cocontinuity yields an initial algebra for the colimit
\cite{fiore2022-quotients-inductive-types}.

\begin{corollary}[Mobiles exist without WISC]
	\label{cor:mobiles-exist}
	In MLTT with quotient types (in the same metatheory as FPS), the QIT of
	infinitary mobiles exists without assuming WISC. Concretely, the colimit
	$Q$ carries an $F$-algebra structure satisfying the permutation equations,
	and this algebra is initial among such algebras.
\end{corollary}

\subsection{Mechanisation status}

We have mechanised in Agda:
(i) the proof that the mobile signature is depth-preserving, and
(ii) the construction of the cocontinuity isomorphism obtained from
depth preservation. The initiality/recursion principle is inherited from
the FPS theorem \cite{fiore2022-quotients-inductive-types}.

\section{Conclusion}

We show that depth-preserving QW types form a class of infinitary quotient
inductive types that admit constructive semantics. This isolates a precise
source of non-constructivity in the FPS construction---unbounded recursive
comparison---and suggests a syntactic criterion for identifying QITs that do not
require choice.

\bibliography{master}

\end{document}
