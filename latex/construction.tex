% Options for packages loaded elsewhere
\PassOptionsToPackage{unicode}{hyperref}
\PassOptionsToPackage{hyphens}{url}
\documentclass[
]{article}
\usepackage{xcolor}
\usepackage{amsmath,amssymb}
\setcounter{secnumdepth}{5}
\usepackage{iftex}
\ifPDFTeX
  \usepackage[T1]{fontenc}
  \usepackage[utf8]{inputenc}
  \usepackage{textcomp} % provide euro and other symbols
\else % if luatex or xetex
  \usepackage{unicode-math} % this also loads fontspec
  \defaultfontfeatures{Scale=MatchLowercase}
  \defaultfontfeatures[\rmfamily]{Ligatures=TeX,Scale=1}
\fi
\usepackage{lmodern}
\ifPDFTeX\else
  % xetex/luatex font selection
\fi
% Use upquote if available, for straight quotes in verbatim environments
\IfFileExists{upquote.sty}{\usepackage{upquote}}{}
\IfFileExists{microtype.sty}{% use microtype if available
  \usepackage[]{microtype}
  \UseMicrotypeSet[protrusion]{basicmath} % disable protrusion for tt fonts
}{}
\makeatletter
\@ifundefined{KOMAClassName}{% if non-KOMA class
  \IfFileExists{parskip.sty}{%
    \usepackage{parskip}
  }{% else
    \setlength{\parindent}{0pt}
    \setlength{\parskip}{6pt plus 2pt minus 1pt}}
}{% if KOMA class
  \KOMAoptions{parskip=half}}
\makeatother
\setlength{\emergencystretch}{3em} % prevent overfull lines
\providecommand{\tightlist}{%
  \setlength{\itemsep}{0pt}\setlength{\parskip}{0pt}}
\usepackage{bookmark}
\IfFileExists{xurl.sty}{\usepackage{xurl}}{} % add URL line breaks if available
\urlstyle{same}
\hypersetup{
  pdftitle={Mobiles as a Quotient-Inductive Type via Plump Ordinals},
  hidelinks,
  pdfcreator={LaTeX via pandoc}}

\title{Mobiles as a Quotient-Inductive Type via Plump Ordinals}
\author{}
\date{}

\begin{document}
\maketitle

{
\setcounter{tocdepth}{3}
\tableofcontents
}
\subsection{Abstract}\label{abstract}

This note describes a construction of infinitary mobiles as a
quotient-inductive type. The construction proceeds in two stages. First,
we give a setoid-based description in terms of a transfinite diagram
indexed by a plump tree ordinal \((T,\le)\). Second, we show that the
same construction can be carried out directly in the category
\textbf{Set}, assuming the availability of non-recursive quotients or
colimit higher inductive types.

The key property of the index order \(T\) is that it possesses a
\emph{definable supremum}: every family of elements indexed by \(I\) has
a canonical upper bound. This avoids the need for WISC (the Weakly
Initial Set of Covers), which is typically required to ensure that
polynomial functors preserve colimits in exact completions. Our
construction is an instance where these general-purpose axioms are
unnecessary.

\subsection{Setoid Infrastructure}\label{setoid-infrastructure}

Before presenting the construction, we establish the fundamental
categorical infrastructure based on setoids.

\subsubsection{Setoids}\label{setoids}

A \textbf{setoid} \(S = (|S|, {\approx_S})\) consists of a set \(|S|\)
equipped with an equivalence relation \({\approx_S}\) satisfying:

\begin{itemize}
\tightlist
\item
  \textbf{Reflexivity}: \(x \approx_S x\) for all \(x \in |S|\)
\item
  \textbf{Symmetry}: \(x \approx_S y\) implies \(y \approx_S x\)
\item
  \textbf{Transitivity}: \(x \approx_S y\) and \(y \approx_S z\) implies
  \(x \approx_S z\)
\end{itemize}

Setoids provide a foundation for working with quotient types while
maintaining computational content.

\subsubsection{Setoid Homomorphisms}\label{setoid-homomorphisms}

A \textbf{setoid homomorphism} \(f: S \to T\) is a function
\(f: |S| \to |T|\) that respects the equivalence relations:

\[x \approx_S y \implies f(x) \approx_T f(y)\]

Setoid homomorphisms compose in the obvious way, and every setoid has an
identity homomorphism.

Two homomorphisms \(f, g : S \to T\) are \textbf{homomorphically
equivalent} if for all \(x \approx_S y\), we have
\(f(x) \approx_T g(y)\). This yields an equivalence relation on the
hom-sets, making setoids into a category \textbf{Setoid}.

\subsubsection{Setoid Isomorphisms}\label{setoid-isomorphisms}

A \textbf{setoid isomorphism} \(\phi: S \cong T\) consists of setoid
homomorphisms \(\phi: S \to T\) and \(\phi^{-1}: T \to S\) such that:

\[\phi^{-1} \circ \phi \approx \mathrm{id}_S \quad \text{and} \quad \phi \circ \phi^{-1} \approx \mathrm{id}_T\]

where \(\approx\) denotes homomorphic equivalence. Setoid isomorphism
defines an equivalence relation on setoids.

\subsubsection{Setoid Functors}\label{setoid-functors}

A \textbf{setoid functor} \(F: \mathbf{Setoid} \to \mathbf{Setoid}\)
assigns to each setoid \(S\) a setoid \(F(S)\), and to each homomorphism
\(f: S \to T\) a homomorphism \(F(f): F(S) \to F(T)\), satisfying:

\begin{itemize}
\tightlist
\item
  \textbf{Identity}: \(F(\mathrm{id}_S) \approx \mathrm{id}_{F(S)}\)
\item
  \textbf{Composition}: \(F(g \circ f) \approx F(g) \circ F(f)\)
\item
  \textbf{Congruence}: \(f \approx g\) implies \(F(f) \approx F(g)\)
\end{itemize}

These conditions ensure that \(F\) is a proper functor on the category
of setoids.

\subsection{Raw infinitary trees and
plumpness}\label{raw-infinitary-trees-and-plumpness}

Fix a set \(I\neq \emptyset\), either finite or infinitary. Define the W
Type \(T\) of \(I\)-branching trees:

\[\mathsf{leaf} : T, \qquad \mathsf{node} : (I \to T) \to T.\] Expressed
as a container, this is: \[
S_{T} = \mathbb{2}; \qquad P_{T}=[\bot,I];\qquad T= W~ S_{T}~ P_{T}
\] We extend this to the plump ordinal trees over the W Type as: \[
S_{Z} = \mathbb{3}; \qquad P_{Z}=[\bot,\mathbb{2} ,I]; \qquad Z=W~S_{Z}~P_{Z}
\] This is because two compare two diagrams over ordinals \(\alpha\) and
\(\beta\), we must be able to join the ordinals. This is necessary for
proving congruence on the reverse direction of cocontinuity. The exact W
type doesn't matter except that it must have sufficient constructors to
be able to cover all branching cases, have a \(\bot\) constructor (so
it's not empty), and a join \(\_\vee\_:Z\to Z \to Z\). This gives rise
to a natural inclusion: \[
\begin{aligned}
\iota_T &:&T&\to &Z&\\
\iota_T&:&sup(\bot,f)&\mapsto &sup(\bot,f)&\\
\iota_T&:&sup(I,f)&\mapsto& sup(I,f)&
\end{aligned}
\] Define the plump order, defined as an inductive-inductive type with
two kinds \(<,\le:Z\to Z\to Prop\) on \(Z\) inductively:

\[
\begin{align*}
\frac{\exists i. \alpha \le f i}{\alpha < sup(s,f) }\qquad\frac{\forall i.f i<\alpha }{sup(s,f)\le \alpha } \qquad \forall i\in b,~s:s_z,~f:p_z~s \to z,~\alpha :z \\
\end{align*}
\]

\subsubsection{Properties of the Plump
Ordering}\label{properties-of-the-plump-ordering}

The plump ordering satisfies the following key properties, all proven
constructively in the Agda formalization:

\textbf{Reflexivity}: For all \(i : Z\), we have \(i \le i\). This is
proven by structural induction:
\[\frac{\forall x. f(x) < \sup(s,f)}{\sup(s,f) \le \sup(s,f)}\]

\textbf{Transitivity}: The plump ordering satisfies multiple forms of
transitivity, proven by mutual recursion: - \(\le\) is transitive:
\(i \le j \land j \le k \implies i \le k\) - \(<\) is transitive:
\(i < j \land j < k \implies i < k\) - Mixed transitivity:
\(i \le j \land j < k \implies i < k\) and
\(i < j \land j \le k \implies i < k\)

\textbf{Well-foundedness}: The strict ordering \(<\) is well-founded,
meaning every element is accessible under the relation. This provides
the foundation for transfinite induction principles over plump trees.

\textbf{Preorder Structure}: The relation \(\le\) forms a preorder on
\(Z\) with reflexivity and transitivity, providing the categorical
foundation for diagram indexing.

\textbf{Subset Characterization}: We can characterize the ordering using
subset relations on the strict predecessors and successors:

For plump trees \(i, j : Z\), we define: - \(i \subseteq j\) (downward
closure) as \(\forall k : Z. k < i \implies k < j\) - \(i \supseteq j\)
(upward closure) as \(\forall k : Z. i < k \implies j < k\)

Intuitively, \(i \subseteq j\) means that every strict predecessor of
\(i\) is also a strict predecessor of \(j\), while \(i \supseteq j\)
means that every strict successor of \(i\) is also a strict successor of
\(j\). Since we're working with well-founded trees rather than classical
sets, ``subset'' here refers to inclusion of the strict predecessor sets
in the well-founded ordering.

\begin{itemize}
\tightlist
\item
  \textbf{Equivalence}: \(i \le j \iff i \subseteq j\) and
  \(i \le j \implies j \supseteq i\)
\end{itemize}

\textbf{Quasi-extensionality}: The plump ordering is quasi-extensional,
meaning:
\[(i \le j \land j \le i) \iff (i \subseteq j \land j \subseteq i)\]

This property connects the ordering structure with the subset structure,
providing a foundation for extensional reasoning about plump trees and
enabling the definable supremum property crucial for cocontinuity.

We then define \(<_T:T\to Z \to Prop\) as
\[t<_T \alpha =\iota_T~ t < \alpha\] This gives us a way to bound trees
by size ordinals. \#\# Mobile Diagram Construction

\subsubsection{Mobile Elements}\label{mobile-elements}

For each plump ordinal \(\alpha : Z\), define the type \(P_0(\alpha)\)
of mobile elements at stage \(\alpha\) inductively:

\[
\begin{align}
\mathsf{leaf} &: \forall \beta : I \to Z. P_0(\sup(I, \beta)) \\
\mathsf{node} &: \forall \beta : I \to Z. \left(\forall i : I. P_0(\beta(i))\right) \to P_0(\sup(I, \beta)) \\
\mathsf{weaken} &: \forall \alpha, \gamma : Z. \alpha \le \gamma \to P_0(\alpha) \to P_0(\gamma)
\end{align}
\]

The key insight is that mobile elements are not simply bounded trees,
but structured terms that can be: - \textbf{Leaves} at any node stage
(indexed by functions \(\beta : I \to Z\)) - \textbf{Nodes} with
recursively defined mobile children at smaller ordinals -
\textbf{Weakened} from smaller to larger stages via the plump ordering

\subsubsection{Mobile Equivalence}\label{mobile-equivalence}

Define the mobile equivalence relation \(\approx^P\) on mobile elements
by the generators:

\begin{itemize}
\tightlist
\item
  \textbf{Leaf equivalence}:
  \(\mathsf{leaf}(\beta) \approx^P \mathsf{leaf}(\gamma)\) for any
  \(\beta, \gamma : I \to Z\)
\item
  \textbf{Node congruence}:
  \(\mathsf{node}(\beta, f) \approx^P \mathsf{node}(\gamma, g)\) if
  \(\forall i : I. f(i) \approx^P g(i)\)
\item
  \textbf{Permutation symmetry}:
  \(\mathsf{node}(\beta, f) \approx^P \mathsf{node}(\beta \circ \pi, f \circ \pi)\)
  for any bijection \(\pi : I \leftrightarrow I\)
\item
  \textbf{Weakening transparency}:
  \(s \approx^P \mathsf{weaken}(\alpha, \gamma, p, s)\) for any
  \(p : \alpha \le \gamma\)
\item
  \textbf{Symmetry and transitivity}: \(\approx^P\) is symmetric and
  transitive
\end{itemize}

This gives mobile equivalence the crucial \textbf{permutation symmetry}
that distinguishes mobile categories from ordinary QIT signatures.

\subsubsection{The Mobile Setoids}\label{the-mobile-setoids}

For each stage \(\alpha : Z\), define the setoid:
\[P(\alpha) := (P_0(\alpha), \approx^P)\]

This construction ensures that each stage forms a proper setoid with
reflexivity, symmetry, and transitivity of the mobile equivalence
relation.

\subsubsection{The Mobile Diagram}\label{the-mobile-diagram}

The mobile diagram structure arises from the plump ordering on ordinals.
For any \(p : \alpha \le \beta\) in the plump order, we obtain a setoid
homomorphism:

\[P(p) : P(\alpha) \to P(\beta)\]

defined by \(P(p)(s) := \mathsf{weaken}(\alpha, \beta, p, s)\) with
congruence established by the weakening transparency of mobile
equivalence.

This yields the mobile diagram:
\[P : (Z, \le) \longrightarrow \mathbf{Setoid}\]

satisfying the functorial properties: - \textbf{Identity}:
\(P(\mathsf{refl}) \approx_h \mathsf{id}\) - \textbf{Composition}:
\(P(q \circ p) \approx_h P(q) \circ P(p)\)

The mobile diagram provides the foundation for constructing colimits
that respect both the ordinal structure and the crucial permutation
symmetries needed for mobile category theory.

\subsubsection{The Mobile Colimit}\label{the-mobile-colimit}

The colimit of the mobile diagram \(P : (Z, \le) \to \mathbf{Setoid}\)
is constructed as follows:

\textbf{Carrier}: The colimit carrier is the dependent sum:
\[\text{Colim}_0 = \sum_{\alpha : Z} |P(\alpha)|\]

\textbf{Equivalence Relation}: Define the colimit equivalence relation
\(\approx^l\) on \(\text{Colim}_0\) inductively by:

\begin{itemize}
\tightlist
\item
  \textbf{Stage equivalence}: \((\alpha, x) \approx^l (\alpha, x')\) if
  \(P(\alpha)[x \approx^P x']\)
\item
  \textbf{Step equivalence}: \((\alpha, x) \approx^l (\beta, P(p)(x))\)
  for any \(p : \alpha \le \beta\)
\item
  \textbf{Symmetry}: \(s \approx^l t \implies t \approx^l s\)
\item
  \textbf{Transitivity}:
  \(s \approx^l t \land t \approx^l u \implies s \approx^l u\)
\end{itemize}

\textbf{The Colimit Setoid}:
\[\text{Colim}(P) := (\text{Colim}_0, \approx^l)\]

\textbf{Cocone Structure}: The canonical injections are:
\[\iota_\alpha : P(\alpha) \to \text{Colim}(P)\]
\[\iota_\alpha(x) := (\alpha, x)\]

These satisfy the cocone property: for any \(p : \alpha \le \beta\), we
have \(\iota_\beta \circ P(p) \approx_h \iota_\alpha\).

\textbf{Universal Property}: \(\text{Colim}(P)\) satisfies the universal
property that for any other cocone with apex \(A\) and injections
\(\{\kappa_\alpha : P(\alpha) \to A\}\), there exists a unique setoid
homomorphism \(h : \text{Colim}(P) \to A\) making all triangles commute.

\subsubsection{The quotient-polynomial
functor}\label{the-quotient-polynomial-functor}

Define an endofunctor on setoids:

\[\widetilde F_B(X,\approx_X) := \bigl( 1 + (I \to X) \bigr) / \approx_F,\]

where \(\approx_F\) is the smallest equivalence relation generated by:

\[\mathsf{Leaf} \approx_F \mathsf{Leaf},\]

\[\mathsf{Node}(f) \approx_F \mathsf{Node}(g) \;\text{if}\; \forall i,\; f(i)\approx_X g(i),\]

\[\mathsf{Node}(f) \approx_F \mathsf{Node}(g) \;\text{if}\; \exists \pi : I\cong I,\ \forall i,\ f(i)\approx_X g(\pi(i)).\]

\subsubsection{Cocontinuity}\label{cocontinuity}

A crucial lemma is that for any \(g : I \to \mathrm{colim}P\), if
\(g(i)\) is represented by \((t_i,x_i)\) then the definable supremum

\[t^* := \mathsf{node}(i \mapsto t_i)\]

satisfies \(t_i \le t^*\), hence

\[g = \iota_{t^*} \circ h\]

for the unique \(h : I \to P(t^*)\) defined by
\(h(i) := P(t_i\le t^*)(x_i)\). This supplies the factorisation needed
to prove that \(\widetilde F_B\) preserves this colimit.

\subsection{Constructing Mobiles Directly in
Set}\label{constructing-mobiles-directly-in-set}

In this section we reinterpret the construction internally in Set, using
a colimit higher inductive type. No use of WISC is required because the
index order \((T,\le)\) has a definable supremum.

\subsubsection{Explicit colimit
constructors}\label{explicit-colimit-constructors}

Define the colimit as a {[}{[}Higher Inductive Type{]}{]} \(M_B\) with
constructors:

\textbf{Points:}

\[\mathsf{inc}(t,x) : M_B \qquad (t : T,\ x : P(t)).\]

\textbf{Paths:}

\[\mathsf{glue}(p,x) : \mathsf{inc}(t,x) = \mathsf{inc}(u, P(p)(x)) \qquad (p : t\le u).\]

No further constructors are required.

\subsubsection{\texorpdfstring{The functor \(\widetilde F_B\) on
Set}{The functor \textbackslash widetilde F\_B on Set}}\label{the-functor-widetilde-f_b-on-set}

Define

\[\widetilde F_B(X) := \left( 1 + (I \to X) \right) / \approx,\]

where \(\approx\) is the same permutation-generated relation as in the
setoid case.

\subsubsection{Cocontinuity in Set}\label{cocontinuity-in-set}

Given \(g : I \to M_B\), write \(g(i)=\mathsf{inc}(t_i,x_i)\). Define
\(t^* := \mathsf{node}(i \mapsto t_i)\) and

\[h(i) := P(t_i\le t^*)(x_i).\]

Then

\[g(i) = \mathsf{inc}(t^*,h(i))\]

up to paths generated by \(\mathsf{glue}\). This provides the factor
needed for the \(\psi\) map of the cocontinuity equivalence.

\subsubsection{The initial algebra}\label{the-initial-algebra}

Cocontinuity yields an isomorphism

\[\widetilde F_B(M_B) \cong \mathrm{Colim}(\widetilde F_B \circ P),\]

from which we obtain a structure map

\[\mathsf{mob}_B : \widetilde F_B(M_B) \to M_B.\]

By transfinite induction along \(T\) and the colimit universal property,
one obtains a unique algebra morphism into any other
\(\widetilde F_B\)-algebra. Hence:

\textbf{Theorem.} \(M_B\) is the {[}{[}Initial Algebra{]}{]} for
\(\widetilde F_B\).

\subsection{Remarks on WISC}\label{remarks-on-wisc}

In general, proving that a polynomial or quotient-polynomial functor
preserves colimits in a quotient completion requires a principle such as
{[}{[}WISC{]}{]} to factor arbitrary maps through a small family of
covers. In this specific case, the definable supremum provided by the
plump order \((T,\le)\) replaces the need for any choice principle.

\end{document}
